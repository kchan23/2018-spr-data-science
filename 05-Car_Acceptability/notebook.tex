
% Default to the notebook output style

    


% Inherit from the specified cell style.




    
\documentclass[11pt]{article}

    
    
    \usepackage[T1]{fontenc}
    % Nicer default font (+ math font) than Computer Modern for most use cases
    \usepackage{mathpazo}

    % Basic figure setup, for now with no caption control since it's done
    % automatically by Pandoc (which extracts ![](path) syntax from Markdown).
    \usepackage{graphicx}
    % We will generate all images so they have a width \maxwidth. This means
    % that they will get their normal width if they fit onto the page, but
    % are scaled down if they would overflow the margins.
    \makeatletter
    \def\maxwidth{\ifdim\Gin@nat@width>\linewidth\linewidth
    \else\Gin@nat@width\fi}
    \makeatother
    \let\Oldincludegraphics\includegraphics
    % Set max figure width to be 80% of text width, for now hardcoded.
    \renewcommand{\includegraphics}[1]{\Oldincludegraphics[width=.8\maxwidth]{#1}}
    % Ensure that by default, figures have no caption (until we provide a
    % proper Figure object with a Caption API and a way to capture that
    % in the conversion process - todo).
    \usepackage{caption}
    \DeclareCaptionLabelFormat{nolabel}{}
    \captionsetup{labelformat=nolabel}

    \usepackage{adjustbox} % Used to constrain images to a maximum size 
    \usepackage{xcolor} % Allow colors to be defined
    \usepackage{enumerate} % Needed for markdown enumerations to work
    \usepackage{geometry} % Used to adjust the document margins
    \usepackage{amsmath} % Equations
    \usepackage{amssymb} % Equations
    \usepackage{textcomp} % defines textquotesingle
    % Hack from http://tex.stackexchange.com/a/47451/13684:
    \AtBeginDocument{%
        \def\PYZsq{\textquotesingle}% Upright quotes in Pygmentized code
    }
    \usepackage{upquote} % Upright quotes for verbatim code
    \usepackage{eurosym} % defines \euro
    \usepackage[mathletters]{ucs} % Extended unicode (utf-8) support
    \usepackage[utf8x]{inputenc} % Allow utf-8 characters in the tex document
    \usepackage{fancyvrb} % verbatim replacement that allows latex
    \usepackage{grffile} % extends the file name processing of package graphics 
                         % to support a larger range 
    % The hyperref package gives us a pdf with properly built
    % internal navigation ('pdf bookmarks' for the table of contents,
    % internal cross-reference links, web links for URLs, etc.)
    \usepackage{hyperref}
    \usepackage{longtable} % longtable support required by pandoc >1.10
    \usepackage{booktabs}  % table support for pandoc > 1.12.2
    \usepackage[inline]{enumitem} % IRkernel/repr support (it uses the enumerate* environment)
    \usepackage[normalem]{ulem} % ulem is needed to support strikethroughs (\sout)
                                % normalem makes italics be italics, not underlines
    

    
    
    % Colors for the hyperref package
    \definecolor{urlcolor}{rgb}{0,.145,.698}
    \definecolor{linkcolor}{rgb}{.71,0.21,0.01}
    \definecolor{citecolor}{rgb}{.12,.54,.11}

    % ANSI colors
    \definecolor{ansi-black}{HTML}{3E424D}
    \definecolor{ansi-black-intense}{HTML}{282C36}
    \definecolor{ansi-red}{HTML}{E75C58}
    \definecolor{ansi-red-intense}{HTML}{B22B31}
    \definecolor{ansi-green}{HTML}{00A250}
    \definecolor{ansi-green-intense}{HTML}{007427}
    \definecolor{ansi-yellow}{HTML}{DDB62B}
    \definecolor{ansi-yellow-intense}{HTML}{B27D12}
    \definecolor{ansi-blue}{HTML}{208FFB}
    \definecolor{ansi-blue-intense}{HTML}{0065CA}
    \definecolor{ansi-magenta}{HTML}{D160C4}
    \definecolor{ansi-magenta-intense}{HTML}{A03196}
    \definecolor{ansi-cyan}{HTML}{60C6C8}
    \definecolor{ansi-cyan-intense}{HTML}{258F8F}
    \definecolor{ansi-white}{HTML}{C5C1B4}
    \definecolor{ansi-white-intense}{HTML}{A1A6B2}

    % commands and environments needed by pandoc snippets
    % extracted from the output of `pandoc -s`
    \providecommand{\tightlist}{%
      \setlength{\itemsep}{0pt}\setlength{\parskip}{0pt}}
    \DefineVerbatimEnvironment{Highlighting}{Verbatim}{commandchars=\\\{\}}
    % Add ',fontsize=\small' for more characters per line
    \newenvironment{Shaded}{}{}
    \newcommand{\KeywordTok}[1]{\textcolor[rgb]{0.00,0.44,0.13}{\textbf{{#1}}}}
    \newcommand{\DataTypeTok}[1]{\textcolor[rgb]{0.56,0.13,0.00}{{#1}}}
    \newcommand{\DecValTok}[1]{\textcolor[rgb]{0.25,0.63,0.44}{{#1}}}
    \newcommand{\BaseNTok}[1]{\textcolor[rgb]{0.25,0.63,0.44}{{#1}}}
    \newcommand{\FloatTok}[1]{\textcolor[rgb]{0.25,0.63,0.44}{{#1}}}
    \newcommand{\CharTok}[1]{\textcolor[rgb]{0.25,0.44,0.63}{{#1}}}
    \newcommand{\StringTok}[1]{\textcolor[rgb]{0.25,0.44,0.63}{{#1}}}
    \newcommand{\CommentTok}[1]{\textcolor[rgb]{0.38,0.63,0.69}{\textit{{#1}}}}
    \newcommand{\OtherTok}[1]{\textcolor[rgb]{0.00,0.44,0.13}{{#1}}}
    \newcommand{\AlertTok}[1]{\textcolor[rgb]{1.00,0.00,0.00}{\textbf{{#1}}}}
    \newcommand{\FunctionTok}[1]{\textcolor[rgb]{0.02,0.16,0.49}{{#1}}}
    \newcommand{\RegionMarkerTok}[1]{{#1}}
    \newcommand{\ErrorTok}[1]{\textcolor[rgb]{1.00,0.00,0.00}{\textbf{{#1}}}}
    \newcommand{\NormalTok}[1]{{#1}}
    
    % Additional commands for more recent versions of Pandoc
    \newcommand{\ConstantTok}[1]{\textcolor[rgb]{0.53,0.00,0.00}{{#1}}}
    \newcommand{\SpecialCharTok}[1]{\textcolor[rgb]{0.25,0.44,0.63}{{#1}}}
    \newcommand{\VerbatimStringTok}[1]{\textcolor[rgb]{0.25,0.44,0.63}{{#1}}}
    \newcommand{\SpecialStringTok}[1]{\textcolor[rgb]{0.73,0.40,0.53}{{#1}}}
    \newcommand{\ImportTok}[1]{{#1}}
    \newcommand{\DocumentationTok}[1]{\textcolor[rgb]{0.73,0.13,0.13}{\textit{{#1}}}}
    \newcommand{\AnnotationTok}[1]{\textcolor[rgb]{0.38,0.63,0.69}{\textbf{\textit{{#1}}}}}
    \newcommand{\CommentVarTok}[1]{\textcolor[rgb]{0.38,0.63,0.69}{\textbf{\textit{{#1}}}}}
    \newcommand{\VariableTok}[1]{\textcolor[rgb]{0.10,0.09,0.49}{{#1}}}
    \newcommand{\ControlFlowTok}[1]{\textcolor[rgb]{0.00,0.44,0.13}{\textbf{{#1}}}}
    \newcommand{\OperatorTok}[1]{\textcolor[rgb]{0.40,0.40,0.40}{{#1}}}
    \newcommand{\BuiltInTok}[1]{{#1}}
    \newcommand{\ExtensionTok}[1]{{#1}}
    \newcommand{\PreprocessorTok}[1]{\textcolor[rgb]{0.74,0.48,0.00}{{#1}}}
    \newcommand{\AttributeTok}[1]{\textcolor[rgb]{0.49,0.56,0.16}{{#1}}}
    \newcommand{\InformationTok}[1]{\textcolor[rgb]{0.38,0.63,0.69}{\textbf{\textit{{#1}}}}}
    \newcommand{\WarningTok}[1]{\textcolor[rgb]{0.38,0.63,0.69}{\textbf{\textit{{#1}}}}}
    
    
    % Define a nice break command that doesn't care if a line doesn't already
    % exist.
    \def\br{\hspace*{\fill} \\* }
    % Math Jax compatability definitions
    \def\gt{>}
    \def\lt{<}
    % Document parameters
    \title{Car\_Acceptability}
    
    
    

    % Pygments definitions
    
\makeatletter
\def\PY@reset{\let\PY@it=\relax \let\PY@bf=\relax%
    \let\PY@ul=\relax \let\PY@tc=\relax%
    \let\PY@bc=\relax \let\PY@ff=\relax}
\def\PY@tok#1{\csname PY@tok@#1\endcsname}
\def\PY@toks#1+{\ifx\relax#1\empty\else%
    \PY@tok{#1}\expandafter\PY@toks\fi}
\def\PY@do#1{\PY@bc{\PY@tc{\PY@ul{%
    \PY@it{\PY@bf{\PY@ff{#1}}}}}}}
\def\PY#1#2{\PY@reset\PY@toks#1+\relax+\PY@do{#2}}

\expandafter\def\csname PY@tok@w\endcsname{\def\PY@tc##1{\textcolor[rgb]{0.73,0.73,0.73}{##1}}}
\expandafter\def\csname PY@tok@c\endcsname{\let\PY@it=\textit\def\PY@tc##1{\textcolor[rgb]{0.25,0.50,0.50}{##1}}}
\expandafter\def\csname PY@tok@cp\endcsname{\def\PY@tc##1{\textcolor[rgb]{0.74,0.48,0.00}{##1}}}
\expandafter\def\csname PY@tok@k\endcsname{\let\PY@bf=\textbf\def\PY@tc##1{\textcolor[rgb]{0.00,0.50,0.00}{##1}}}
\expandafter\def\csname PY@tok@kp\endcsname{\def\PY@tc##1{\textcolor[rgb]{0.00,0.50,0.00}{##1}}}
\expandafter\def\csname PY@tok@kt\endcsname{\def\PY@tc##1{\textcolor[rgb]{0.69,0.00,0.25}{##1}}}
\expandafter\def\csname PY@tok@o\endcsname{\def\PY@tc##1{\textcolor[rgb]{0.40,0.40,0.40}{##1}}}
\expandafter\def\csname PY@tok@ow\endcsname{\let\PY@bf=\textbf\def\PY@tc##1{\textcolor[rgb]{0.67,0.13,1.00}{##1}}}
\expandafter\def\csname PY@tok@nb\endcsname{\def\PY@tc##1{\textcolor[rgb]{0.00,0.50,0.00}{##1}}}
\expandafter\def\csname PY@tok@nf\endcsname{\def\PY@tc##1{\textcolor[rgb]{0.00,0.00,1.00}{##1}}}
\expandafter\def\csname PY@tok@nc\endcsname{\let\PY@bf=\textbf\def\PY@tc##1{\textcolor[rgb]{0.00,0.00,1.00}{##1}}}
\expandafter\def\csname PY@tok@nn\endcsname{\let\PY@bf=\textbf\def\PY@tc##1{\textcolor[rgb]{0.00,0.00,1.00}{##1}}}
\expandafter\def\csname PY@tok@ne\endcsname{\let\PY@bf=\textbf\def\PY@tc##1{\textcolor[rgb]{0.82,0.25,0.23}{##1}}}
\expandafter\def\csname PY@tok@nv\endcsname{\def\PY@tc##1{\textcolor[rgb]{0.10,0.09,0.49}{##1}}}
\expandafter\def\csname PY@tok@no\endcsname{\def\PY@tc##1{\textcolor[rgb]{0.53,0.00,0.00}{##1}}}
\expandafter\def\csname PY@tok@nl\endcsname{\def\PY@tc##1{\textcolor[rgb]{0.63,0.63,0.00}{##1}}}
\expandafter\def\csname PY@tok@ni\endcsname{\let\PY@bf=\textbf\def\PY@tc##1{\textcolor[rgb]{0.60,0.60,0.60}{##1}}}
\expandafter\def\csname PY@tok@na\endcsname{\def\PY@tc##1{\textcolor[rgb]{0.49,0.56,0.16}{##1}}}
\expandafter\def\csname PY@tok@nt\endcsname{\let\PY@bf=\textbf\def\PY@tc##1{\textcolor[rgb]{0.00,0.50,0.00}{##1}}}
\expandafter\def\csname PY@tok@nd\endcsname{\def\PY@tc##1{\textcolor[rgb]{0.67,0.13,1.00}{##1}}}
\expandafter\def\csname PY@tok@s\endcsname{\def\PY@tc##1{\textcolor[rgb]{0.73,0.13,0.13}{##1}}}
\expandafter\def\csname PY@tok@sd\endcsname{\let\PY@it=\textit\def\PY@tc##1{\textcolor[rgb]{0.73,0.13,0.13}{##1}}}
\expandafter\def\csname PY@tok@si\endcsname{\let\PY@bf=\textbf\def\PY@tc##1{\textcolor[rgb]{0.73,0.40,0.53}{##1}}}
\expandafter\def\csname PY@tok@se\endcsname{\let\PY@bf=\textbf\def\PY@tc##1{\textcolor[rgb]{0.73,0.40,0.13}{##1}}}
\expandafter\def\csname PY@tok@sr\endcsname{\def\PY@tc##1{\textcolor[rgb]{0.73,0.40,0.53}{##1}}}
\expandafter\def\csname PY@tok@ss\endcsname{\def\PY@tc##1{\textcolor[rgb]{0.10,0.09,0.49}{##1}}}
\expandafter\def\csname PY@tok@sx\endcsname{\def\PY@tc##1{\textcolor[rgb]{0.00,0.50,0.00}{##1}}}
\expandafter\def\csname PY@tok@m\endcsname{\def\PY@tc##1{\textcolor[rgb]{0.40,0.40,0.40}{##1}}}
\expandafter\def\csname PY@tok@gh\endcsname{\let\PY@bf=\textbf\def\PY@tc##1{\textcolor[rgb]{0.00,0.00,0.50}{##1}}}
\expandafter\def\csname PY@tok@gu\endcsname{\let\PY@bf=\textbf\def\PY@tc##1{\textcolor[rgb]{0.50,0.00,0.50}{##1}}}
\expandafter\def\csname PY@tok@gd\endcsname{\def\PY@tc##1{\textcolor[rgb]{0.63,0.00,0.00}{##1}}}
\expandafter\def\csname PY@tok@gi\endcsname{\def\PY@tc##1{\textcolor[rgb]{0.00,0.63,0.00}{##1}}}
\expandafter\def\csname PY@tok@gr\endcsname{\def\PY@tc##1{\textcolor[rgb]{1.00,0.00,0.00}{##1}}}
\expandafter\def\csname PY@tok@ge\endcsname{\let\PY@it=\textit}
\expandafter\def\csname PY@tok@gs\endcsname{\let\PY@bf=\textbf}
\expandafter\def\csname PY@tok@gp\endcsname{\let\PY@bf=\textbf\def\PY@tc##1{\textcolor[rgb]{0.00,0.00,0.50}{##1}}}
\expandafter\def\csname PY@tok@go\endcsname{\def\PY@tc##1{\textcolor[rgb]{0.53,0.53,0.53}{##1}}}
\expandafter\def\csname PY@tok@gt\endcsname{\def\PY@tc##1{\textcolor[rgb]{0.00,0.27,0.87}{##1}}}
\expandafter\def\csname PY@tok@err\endcsname{\def\PY@bc##1{\setlength{\fboxsep}{0pt}\fcolorbox[rgb]{1.00,0.00,0.00}{1,1,1}{\strut ##1}}}
\expandafter\def\csname PY@tok@kc\endcsname{\let\PY@bf=\textbf\def\PY@tc##1{\textcolor[rgb]{0.00,0.50,0.00}{##1}}}
\expandafter\def\csname PY@tok@kd\endcsname{\let\PY@bf=\textbf\def\PY@tc##1{\textcolor[rgb]{0.00,0.50,0.00}{##1}}}
\expandafter\def\csname PY@tok@kn\endcsname{\let\PY@bf=\textbf\def\PY@tc##1{\textcolor[rgb]{0.00,0.50,0.00}{##1}}}
\expandafter\def\csname PY@tok@kr\endcsname{\let\PY@bf=\textbf\def\PY@tc##1{\textcolor[rgb]{0.00,0.50,0.00}{##1}}}
\expandafter\def\csname PY@tok@bp\endcsname{\def\PY@tc##1{\textcolor[rgb]{0.00,0.50,0.00}{##1}}}
\expandafter\def\csname PY@tok@fm\endcsname{\def\PY@tc##1{\textcolor[rgb]{0.00,0.00,1.00}{##1}}}
\expandafter\def\csname PY@tok@vc\endcsname{\def\PY@tc##1{\textcolor[rgb]{0.10,0.09,0.49}{##1}}}
\expandafter\def\csname PY@tok@vg\endcsname{\def\PY@tc##1{\textcolor[rgb]{0.10,0.09,0.49}{##1}}}
\expandafter\def\csname PY@tok@vi\endcsname{\def\PY@tc##1{\textcolor[rgb]{0.10,0.09,0.49}{##1}}}
\expandafter\def\csname PY@tok@vm\endcsname{\def\PY@tc##1{\textcolor[rgb]{0.10,0.09,0.49}{##1}}}
\expandafter\def\csname PY@tok@sa\endcsname{\def\PY@tc##1{\textcolor[rgb]{0.73,0.13,0.13}{##1}}}
\expandafter\def\csname PY@tok@sb\endcsname{\def\PY@tc##1{\textcolor[rgb]{0.73,0.13,0.13}{##1}}}
\expandafter\def\csname PY@tok@sc\endcsname{\def\PY@tc##1{\textcolor[rgb]{0.73,0.13,0.13}{##1}}}
\expandafter\def\csname PY@tok@dl\endcsname{\def\PY@tc##1{\textcolor[rgb]{0.73,0.13,0.13}{##1}}}
\expandafter\def\csname PY@tok@s2\endcsname{\def\PY@tc##1{\textcolor[rgb]{0.73,0.13,0.13}{##1}}}
\expandafter\def\csname PY@tok@sh\endcsname{\def\PY@tc##1{\textcolor[rgb]{0.73,0.13,0.13}{##1}}}
\expandafter\def\csname PY@tok@s1\endcsname{\def\PY@tc##1{\textcolor[rgb]{0.73,0.13,0.13}{##1}}}
\expandafter\def\csname PY@tok@mb\endcsname{\def\PY@tc##1{\textcolor[rgb]{0.40,0.40,0.40}{##1}}}
\expandafter\def\csname PY@tok@mf\endcsname{\def\PY@tc##1{\textcolor[rgb]{0.40,0.40,0.40}{##1}}}
\expandafter\def\csname PY@tok@mh\endcsname{\def\PY@tc##1{\textcolor[rgb]{0.40,0.40,0.40}{##1}}}
\expandafter\def\csname PY@tok@mi\endcsname{\def\PY@tc##1{\textcolor[rgb]{0.40,0.40,0.40}{##1}}}
\expandafter\def\csname PY@tok@il\endcsname{\def\PY@tc##1{\textcolor[rgb]{0.40,0.40,0.40}{##1}}}
\expandafter\def\csname PY@tok@mo\endcsname{\def\PY@tc##1{\textcolor[rgb]{0.40,0.40,0.40}{##1}}}
\expandafter\def\csname PY@tok@ch\endcsname{\let\PY@it=\textit\def\PY@tc##1{\textcolor[rgb]{0.25,0.50,0.50}{##1}}}
\expandafter\def\csname PY@tok@cm\endcsname{\let\PY@it=\textit\def\PY@tc##1{\textcolor[rgb]{0.25,0.50,0.50}{##1}}}
\expandafter\def\csname PY@tok@cpf\endcsname{\let\PY@it=\textit\def\PY@tc##1{\textcolor[rgb]{0.25,0.50,0.50}{##1}}}
\expandafter\def\csname PY@tok@c1\endcsname{\let\PY@it=\textit\def\PY@tc##1{\textcolor[rgb]{0.25,0.50,0.50}{##1}}}
\expandafter\def\csname PY@tok@cs\endcsname{\let\PY@it=\textit\def\PY@tc##1{\textcolor[rgb]{0.25,0.50,0.50}{##1}}}

\def\PYZbs{\char`\\}
\def\PYZus{\char`\_}
\def\PYZob{\char`\{}
\def\PYZcb{\char`\}}
\def\PYZca{\char`\^}
\def\PYZam{\char`\&}
\def\PYZlt{\char`\<}
\def\PYZgt{\char`\>}
\def\PYZsh{\char`\#}
\def\PYZpc{\char`\%}
\def\PYZdl{\char`\$}
\def\PYZhy{\char`\-}
\def\PYZsq{\char`\'}
\def\PYZdq{\char`\"}
\def\PYZti{\char`\~}
% for compatibility with earlier versions
\def\PYZat{@}
\def\PYZlb{[}
\def\PYZrb{]}
\makeatother


    % Exact colors from NB
    \definecolor{incolor}{rgb}{0.0, 0.0, 0.5}
    \definecolor{outcolor}{rgb}{0.545, 0.0, 0.0}



    
    % Prevent overflowing lines due to hard-to-break entities
    \sloppy 
    % Setup hyperref package
    \hypersetup{
      breaklinks=true,  % so long urls are correctly broken across lines
      colorlinks=true,
      urlcolor=urlcolor,
      linkcolor=linkcolor,
      citecolor=citecolor,
      }
    % Slightly bigger margins than the latex defaults
    
    \geometry{verbose,tmargin=1in,bmargin=1in,lmargin=1in,rmargin=1in}
    
    

    \begin{document}
    
    
    \maketitle
    
    

    
    \begin{Verbatim}[commandchars=\\\{\}]
{\color{incolor}In [{\color{incolor}14}]:} car \PY{o}{=} read.csv\PY{p}{(}\PY{l+s}{\PYZsq{}}\PY{l+s}{car.data.txt\PYZsq{}}\PY{p}{,} header \PY{o}{=} \PY{n+nb+bp}{F}\PY{p}{)}
         \PY{k+kp}{colnames}\PY{p}{(}car\PY{p}{)} \PY{o}{=} \PY{k+kt}{c}\PY{p}{(}\PY{l+s}{\PYZsq{}}\PY{l+s}{buying\PYZus{}price\PYZsq{}}\PY{p}{,} \PY{l+s}{\PYZsq{}}\PY{l+s}{maint\PYZus{}price\PYZsq{}}\PY{p}{,} \PY{l+s}{\PYZsq{}}\PY{l+s}{no\PYZus{}doors\PYZsq{}}\PY{p}{,} \PY{l+s}{\PYZsq{}}\PY{l+s}{capacity\PYZsq{}}\PY{p}{,} \PY{l+s}{\PYZsq{}}\PY{l+s}{lug\PYZus{}boot\PYZsq{}}\PY{p}{,} \PY{l+s}{\PYZsq{}}\PY{l+s}{safety\PYZsq{}}\PY{p}{,} \PY{l+s}{\PYZsq{}}\PY{l+s}{acc\PYZsq{}}\PY{p}{)}
\end{Verbatim}


    Why is it a best practice to run the next cell right after loading data?

     Checking out the dimensions of the data allows you to see whether the
data is given to you as a vector, matrix, or array. Although data
frames/matrices will be most common when pulling data sets, it is good
to check in order to understand how the data is presented to you.

In the context of a dataframe, the dim gives you the dimensions in (row,
column) format, which usually corresponds to (\#observations,
\#features).

    \begin{Verbatim}[commandchars=\\\{\}]
{\color{incolor}In [{\color{incolor}15}]:} \PY{k+kp}{dim}\PY{p}{(}car\PY{p}{)}
\end{Verbatim}


    \begin{enumerate*}
\item 1728
\item 7
\end{enumerate*}


    
    Why are \texttt{no\_doors} and \texttt{capacity} factors and not
integers?

    \texttt{no\_doors} and \texttt{capacity} are factors because they are
defined in a categorical context where there is a ``more'' term that
encapsules anything above a certain number for their respective
features. Also notice that they only count common values of each feature
signifying that each observation is going to fall under a category.

    \begin{Verbatim}[commandchars=\\\{\}]
{\color{incolor}In [{\color{incolor}16}]:} str\PY{p}{(}car\PY{p}{)}
\end{Verbatim}


    \begin{Verbatim}[commandchars=\\\{\}]
'data.frame':	1728 obs. of  7 variables:
 \$ buying\_price: Factor w/ 4 levels "high","low","med",..: 4 4 4 4 4 4 4 4 4 4 {\ldots}
 \$ maint\_price : Factor w/ 4 levels "high","low","med",..: 4 4 4 4 4 4 4 4 4 4 {\ldots}
 \$ no\_doors    : Factor w/ 4 levels "2","3","4","5more": 1 1 1 1 1 1 1 1 1 1 {\ldots}
 \$ capacity    : Factor w/ 3 levels "2","4","more": 1 1 1 1 1 1 1 1 1 2 {\ldots}
 \$ lug\_boot    : Factor w/ 3 levels "big","med","small": 3 3 3 2 2 2 1 1 1 3 {\ldots}
 \$ safety      : Factor w/ 3 levels "high","low","med": 2 3 1 2 3 1 2 3 1 2 {\ldots}
 \$ acc         : Factor w/ 4 levels "acc","good","unacc",..: 3 3 3 3 3 3 3 3 3 3 {\ldots}

    \end{Verbatim}

    What do the commands in the next cell do? Why are they necessary?

    \texttt{library(caret)} loads the caret package, which is necessary for
functions used later in the notebook. One such function used later is
the \texttt{createDataPartition} function.

\texttt{set.seed} is used so that others can replicate the results from
your work via a pseudo random number generator.

    \begin{Verbatim}[commandchars=\\\{\}]
{\color{incolor}In [{\color{incolor}17}]:} \PY{k+kn}{library}\PY{p}{(}caret\PY{p}{)}
         \PY{k+kp}{set.seed}\PY{p}{(}\PY{l+m}{10}\PY{p}{)}
\end{Verbatim}


    What does the next cell show? \textbf{Don't interpret the results. Tell
me what the results are.}

    The next cell shows that for the \texttt{acc} feature in the dataset,
there are: - 384 observations under acc - 69 observations under good -
1210 observations under unacc - 65 observations under vgood

    \begin{Verbatim}[commandchars=\\\{\}]
{\color{incolor}In [{\color{incolor}18}]:} \PY{k+kp}{table}\PY{p}{(}car\PY{o}{\PYZdl{}}acc\PY{p}{)}
\end{Verbatim}


    
    \begin{verbatim}

  acc  good unacc vgood 
  384    69  1210    65 
    \end{verbatim}

    
    \begin{Verbatim}[commandchars=\\\{\}]
{\color{incolor}In [{\color{incolor}19}]:} \PY{k+kp}{prop.table}\PY{p}{(}\PY{k+kp}{table}\PY{p}{(}car\PY{o}{\PYZdl{}}acc\PY{p}{)}\PY{p}{)}
\end{Verbatim}


    
    \begin{verbatim}

       acc       good      unacc      vgood 
0.22222222 0.03993056 0.70023148 0.03761574 
    \end{verbatim}

    
    If \texttt{acc} is the target, what kind of a machine learning problem
is this? What would be a good metric for the problem? Why? What would be
a good benchmark? Why?

    Since we have a categorical target with known labels, we can identify
this as a classification problem. We can create a train-test split to
fit a model and use the mean average error(MAE) between the actual and
predicted values as a metric.

*Note: Upon further reading, it seems that accuracy may not be the best
measure due to the fact that the observations are heavily unbalanced.
Precision or recall may be better depending on whether you want to
minimize false positives or false negatives.

    \begin{Verbatim}[commandchars=\\\{\}]
{\color{incolor}In [{\color{incolor}20}]:} \PY{k+kn}{library}\PY{p}{(}repr\PY{p}{)}
         \PY{k+kp}{options}\PY{p}{(}repr.plot.width\PY{o}{=}\PY{l+m}{10}\PY{p}{,} repr.plot.height\PY{o}{=}\PY{l+m}{4}\PY{p}{)}
         barplot\PY{p}{(}\PY{k+kp}{prop.table}\PY{p}{(}\PY{k+kp}{table}\PY{p}{(}car\PY{o}{\PYZdl{}}acc\PY{p}{)}\PY{p}{)}\PY{p}{)}
\end{Verbatim}


    \begin{center}
    \adjustimage{max size={0.9\linewidth}{0.9\paperheight}}{output_16_0.png}
    \end{center}
    { \hspace*{\fill} \\}
    
    Describe the distribution of the target class.

     \texttt{good} and \texttt{vgood} have very low frequencies whereas
\texttt{unacc} dwarfs the rest of the categories.

    What does the next cell show?

     The cell below takes the sum of null values in each column. This is
useful for checking for missing data on each feature.

    \begin{Verbatim}[commandchars=\\\{\}]
{\color{incolor}In [{\color{incolor}21}]:} \PY{k+kp}{colSums}\PY{p}{(}\PY{k+kp}{is.na}\PY{p}{(}car\PY{p}{)}\PY{p}{)}
\end{Verbatim}


    \begin{description*}
\item[buying\textbackslash{}\_price] 0
\item[maint\textbackslash{}\_price] 0
\item[no\textbackslash{}\_doors] 0
\item[capacity] 0
\item[lug\textbackslash{}\_boot] 0
\item[safety] 0
\item[acc] 0
\end{description*}


    
    \begin{Verbatim}[commandchars=\\\{\}]
{\color{incolor}In [{\color{incolor}22}]:} car\PYZus{}numeric\PYZus{}features \PY{o}{=} \PY{k+kp}{Filter}\PY{p}{(}\PY{k+kp}{is.numeric}\PY{p}{,} car\PY{p}{)}
\end{Verbatim}


    What do the commands in the next cell do? Why are they necessary?

     A train-test split is necessary when attempting to do predictions to
avoid overfitting when you try to evaluate a model with the dataset you
used to create the model.

\begin{itemize}
\tightlist
\item
  train\_index stores the indices of each observation from the original
  dataset that has been subsetted as the training set.
\item
  training\_data stores the observations by referencing
  \texttt{train\_index} to specify the training data
\item
  test\_data uses syntax from R that subsets the \texttt{car} data frame
  via dropping rows that are indexed in \texttt{train\_index}
\end{itemize}

    \begin{Verbatim}[commandchars=\\\{\}]
{\color{incolor}In [{\color{incolor}23}]:} train\PYZus{}index \PY{o}{\PYZlt{}\PYZhy{}} createDataPartition\PY{p}{(}car\PY{o}{\PYZdl{}}acc\PY{p}{,}p\PY{o}{=}\PY{l+m}{0.8}\PY{p}{,}\PY{k+kt}{list}\PY{o}{=}\PY{k+kc}{FALSE}\PY{p}{)}
         training\PYZus{}data \PY{o}{\PYZlt{}\PYZhy{}} car\PY{p}{[}train\PYZus{}index\PY{p}{,}\PY{p}{]}
         test\PYZus{}data \PY{o}{\PYZlt{}\PYZhy{}} car\PY{p}{[}\PY{o}{\PYZhy{}}train\PYZus{}index\PY{p}{,}\PY{p}{]}
\end{Verbatim}


    Interpret these results. How should these results impact our work?

     The difference in proportions for each category is in the thousandths
or less, showing that the training and test data are fairly
representative of the overall data. The results from this split should
therefore not create much error via sample variance on the train/test
data.

    \begin{Verbatim}[commandchars=\\\{\}]
{\color{incolor}In [{\color{incolor}48}]:} \PY{k+kp}{prop.table}\PY{p}{(}\PY{k+kp}{table}\PY{p}{(}car\PY{o}{\PYZdl{}}acc\PY{p}{)}\PY{p}{)}
         \PY{k+kp}{prop.table}\PY{p}{(}\PY{k+kp}{table}\PY{p}{(}training\PYZus{}data\PY{o}{\PYZdl{}}acc\PY{p}{)}\PY{p}{)}
         \PY{k+kp}{prop.table}\PY{p}{(}\PY{k+kp}{table}\PY{p}{(}test\PYZus{}data\PY{o}{\PYZdl{}}acc\PY{p}{)}\PY{p}{)}
\end{Verbatim}


    
    \begin{verbatim}

       acc       good      unacc      vgood 
0.22222222 0.03993056 0.70023148 0.03761574 
    \end{verbatim}

    
    
    \begin{verbatim}

       acc       good      unacc      vgood 
0.22254335 0.04046243 0.69942197 0.03757225 
    \end{verbatim}

    
    
    \begin{verbatim}

      acc      good     unacc     vgood 
0.2209302 0.0377907 0.7034884 0.0377907 
    \end{verbatim}

    
    \begin{Verbatim}[commandchars=\\\{\}]
{\color{incolor}In [{\color{incolor}31}]:} prop\PYZus{}all \PY{o}{=} \PY{k+kp}{prop.table}\PY{p}{(}\PY{k+kp}{table}\PY{p}{(}car\PY{o}{\PYZdl{}}acc\PY{p}{)}\PY{p}{)}
         prop\PYZus{}test \PY{o}{=} \PY{k+kp}{prop.table}\PY{p}{(}\PY{k+kp}{table}\PY{p}{(}test\PYZus{}data\PY{o}{\PYZdl{}}acc\PY{p}{)}\PY{p}{)}
         prop\PYZus{}train \PY{o}{=} \PY{k+kp}{prop.table}\PY{p}{(}\PY{k+kp}{table}\PY{p}{(}training\PYZus{}data\PY{o}{\PYZdl{}}acc\PY{p}{)}\PY{p}{)}
\end{Verbatim}


    \begin{Verbatim}[commandchars=\\\{\}]
{\color{incolor}In [{\color{incolor}47}]:} \PY{k+kp}{noquote}\PY{p}{(}\PY{l+s}{\PYZsq{}}\PY{l+s}{Proportion Difference(Test)\PYZsq{}}\PY{p}{)}
         prop\PYZus{}all\PY{o}{\PYZhy{}}prop\PYZus{}test
         \PY{k+kp}{noquote}\PY{p}{(}\PY{l+s}{\PYZsq{}}\PY{l+s}{Proportion Difference(Train)\PYZsq{}}\PY{p}{)}
         prop\PYZus{}all\PY{o}{\PYZhy{}}prop\PYZus{}train
\end{Verbatim}


    
    \begin{verbatim}
[1] Proportion Difference(Test)
    \end{verbatim}

    
    
    \begin{verbatim}

          acc          good         unacc         vgood 
 0.0012919897  0.0021398579 -0.0032568906 -0.0001749569 
    \end{verbatim}

    
    
    \begin{verbatim}
[1] Proportion Difference(Train)
    \end{verbatim}

    
    
    \begin{verbatim}

          acc          good         unacc         vgood 
-3.211304e-04 -5.318722e-04  8.095162e-04  4.348641e-05 
    \end{verbatim}

    
    \hypertarget{eda}{%
\subsection{EDA}\label{eda}}

    \begin{Verbatim}[commandchars=\\\{\}]
{\color{incolor}In [{\color{incolor}25}]:} \PY{k+kn}{attach}\PY{p}{(}training\PYZus{}data\PY{p}{)}
\end{Verbatim}


    \begin{Verbatim}[commandchars=\\\{\}]
The following objects are masked from training\_data (pos = 3):

    acc, buying\_price, capacity, lug\_boot, maint\_price, no\_doors,
    safety


    \end{Verbatim}

    \begin{Verbatim}[commandchars=\\\{\}]
{\color{incolor}In [{\color{incolor}26}]:} \PY{k+kp}{table}\PY{p}{(}buying\PYZus{}price\PY{p}{)}
         \PY{k+kp}{prop.table}\PY{p}{(}\PY{k+kp}{table}\PY{p}{(}buying\PYZus{}price\PY{p}{,} acc\PY{p}{)}\PY{p}{,} \PY{l+m}{1}\PY{p}{)}
         
         \PY{k+kp}{table}\PY{p}{(}maint\PYZus{}price\PY{p}{)}
         \PY{k+kp}{prop.table}\PY{p}{(}\PY{k+kp}{table}\PY{p}{(}maint\PYZus{}price\PY{p}{,} acc\PY{p}{)}\PY{p}{,} \PY{l+m}{1}\PY{p}{)}
         
         \PY{k+kp}{table}\PY{p}{(}no\PYZus{}doors\PY{p}{)}
         \PY{k+kp}{prop.table}\PY{p}{(}\PY{k+kp}{table}\PY{p}{(}no\PYZus{}doors\PY{p}{,} acc\PY{p}{)}\PY{p}{,} \PY{l+m}{1}\PY{p}{)}
         
         \PY{k+kp}{table}\PY{p}{(}capacity\PY{p}{)}
         \PY{k+kp}{prop.table}\PY{p}{(}\PY{k+kp}{table}\PY{p}{(}capacity\PY{p}{,} acc\PY{p}{)}\PY{p}{,} \PY{l+m}{1}\PY{p}{)}
         
         \PY{k+kp}{table}\PY{p}{(}lug\PYZus{}boot\PY{p}{)}
         \PY{k+kp}{prop.table}\PY{p}{(}\PY{k+kp}{table}\PY{p}{(}lug\PYZus{}boot\PY{p}{,} acc\PY{p}{)}\PY{p}{,} \PY{l+m}{1}\PY{p}{)}
         
         \PY{k+kp}{table}\PY{p}{(}safety\PY{p}{)}
         \PY{k+kp}{prop.table}\PY{p}{(}\PY{k+kp}{table}\PY{p}{(}safety\PY{p}{,} acc\PY{p}{)}\PY{p}{,} \PY{l+m}{1}\PY{p}{)}
\end{Verbatim}


    
    \begin{verbatim}
buying_price
 high   low   med vhigh 
  343   347   346   348 
    \end{verbatim}

    
    
    \begin{verbatim}
            acc
buying_price        acc       good      unacc      vgood
       high  0.26239067 0.00000000 0.73760933 0.00000000
       low   0.20172911 0.11239193 0.59365994 0.09221902
       med   0.26589595 0.04913295 0.62716763 0.05780347
       vhigh 0.16091954 0.00000000 0.83908046 0.00000000
    \end{verbatim}

    
    
    \begin{verbatim}
maint_price
 high   low   med vhigh 
  346   352   339   347 
    \end{verbatim}

    
    
    \begin{verbatim}
           acc
maint_price        acc       good      unacc      vgood
      high  0.24277457 0.00000000 0.72543353 0.03179191
      low   0.21022727 0.10511364 0.62500000 0.05965909
      med   0.27433628 0.05604720 0.61061947 0.05899705
      vhigh 0.16426513 0.00000000 0.83573487 0.00000000
    \end{verbatim}

    
    
    \begin{verbatim}
no_doors
    2     3     4 5more 
  345   350   345   344 
    \end{verbatim}

    
    
    \begin{verbatim}
        acc
no_doors        acc       good      unacc      vgood
   2     0.16231884 0.03768116 0.77101449 0.02898551
   3     0.23714286 0.04000000 0.68857143 0.03428571
   4     0.25507246 0.04637681 0.65507246 0.04347826
   5more 0.23546512 0.03779070 0.68313953 0.04360465
    \end{verbatim}

    
    
    \begin{verbatim}
capacity
   2    4 more 
 470  467  447 
    \end{verbatim}

    
    
    \begin{verbatim}
        acc
capacity        acc       good      unacc      vgood
    2    0.00000000 0.00000000 1.00000000 0.00000000
    4    0.32334047 0.06638116 0.55674518 0.05353319
    more 0.35123043 0.05592841 0.53243848 0.06040268
    \end{verbatim}

    
    
    \begin{verbatim}
lug_boot
  big   med small 
  461   465   458 
    \end{verbatim}

    
    
    \begin{verbatim}
        acc
lug_boot        acc       good      unacc      vgood
   big   0.24728850 0.04772234 0.63557484 0.06941432
   med   0.23655914 0.03870968 0.68172043 0.04301075
   small 0.18340611 0.03493450 0.78165939 0.00000000
    \end{verbatim}

    
    
    \begin{verbatim}
safety
high  low  med 
 463  450  471 
    \end{verbatim}

    
    
    \begin{verbatim}
      acc
safety        acc       good      unacc      vgood
  high 0.34557235 0.04967603 0.49244060 0.11231102
  low  0.00000000 0.00000000 1.00000000 0.00000000
  med  0.31422505 0.07006369 0.61571125 0.00000000
    \end{verbatim}

    
    Which two features will be the strongest indicator of unacceptability?
Why?

    \texttt{safety} and \texttt{capacity} will be the strongest indicator of
unacceptability since they each have a factor that fully predicts when a
car is unnacceptable (low safety, 2-seater)

    \hypertarget{bonus}{%
\subsection{BONUS}\label{bonus}}

Prepare a vectorized model for acceptability.


    % Add a bibliography block to the postdoc
    
    
    
    \end{document}
