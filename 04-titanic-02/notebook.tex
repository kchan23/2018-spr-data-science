
% Default to the notebook output style

    


% Inherit from the specified cell style.




    
\documentclass[11pt]{article}

    
    
    \usepackage[T1]{fontenc}
    % Nicer default font (+ math font) than Computer Modern for most use cases
    \usepackage{mathpazo}

    % Basic figure setup, for now with no caption control since it's done
    % automatically by Pandoc (which extracts ![](path) syntax from Markdown).
    \usepackage{graphicx}
    % We will generate all images so they have a width \maxwidth. This means
    % that they will get their normal width if they fit onto the page, but
    % are scaled down if they would overflow the margins.
    \makeatletter
    \def\maxwidth{\ifdim\Gin@nat@width>\linewidth\linewidth
    \else\Gin@nat@width\fi}
    \makeatother
    \let\Oldincludegraphics\includegraphics
    % Set max figure width to be 80% of text width, for now hardcoded.
    \renewcommand{\includegraphics}[1]{\Oldincludegraphics[width=.8\maxwidth]{#1}}
    % Ensure that by default, figures have no caption (until we provide a
    % proper Figure object with a Caption API and a way to capture that
    % in the conversion process - todo).
    \usepackage{caption}
    \DeclareCaptionLabelFormat{nolabel}{}
    \captionsetup{labelformat=nolabel}

    \usepackage{adjustbox} % Used to constrain images to a maximum size 
    \usepackage{xcolor} % Allow colors to be defined
    \usepackage{enumerate} % Needed for markdown enumerations to work
    \usepackage{geometry} % Used to adjust the document margins
    \usepackage{amsmath} % Equations
    \usepackage{amssymb} % Equations
    \usepackage{textcomp} % defines textquotesingle
    % Hack from http://tex.stackexchange.com/a/47451/13684:
    \AtBeginDocument{%
        \def\PYZsq{\textquotesingle}% Upright quotes in Pygmentized code
    }
    \usepackage{upquote} % Upright quotes for verbatim code
    \usepackage{eurosym} % defines \euro
    \usepackage[mathletters]{ucs} % Extended unicode (utf-8) support
    \usepackage[utf8x]{inputenc} % Allow utf-8 characters in the tex document
    \usepackage{fancyvrb} % verbatim replacement that allows latex
    \usepackage{grffile} % extends the file name processing of package graphics 
                         % to support a larger range 
    % The hyperref package gives us a pdf with properly built
    % internal navigation ('pdf bookmarks' for the table of contents,
    % internal cross-reference links, web links for URLs, etc.)
    \usepackage{hyperref}
    \usepackage{longtable} % longtable support required by pandoc >1.10
    \usepackage{booktabs}  % table support for pandoc > 1.12.2
    \usepackage[inline]{enumitem} % IRkernel/repr support (it uses the enumerate* environment)
    \usepackage[normalem]{ulem} % ulem is needed to support strikethroughs (\sout)
                                % normalem makes italics be italics, not underlines
    

    
    
    % Colors for the hyperref package
    \definecolor{urlcolor}{rgb}{0,.145,.698}
    \definecolor{linkcolor}{rgb}{.71,0.21,0.01}
    \definecolor{citecolor}{rgb}{.12,.54,.11}

    % ANSI colors
    \definecolor{ansi-black}{HTML}{3E424D}
    \definecolor{ansi-black-intense}{HTML}{282C36}
    \definecolor{ansi-red}{HTML}{E75C58}
    \definecolor{ansi-red-intense}{HTML}{B22B31}
    \definecolor{ansi-green}{HTML}{00A250}
    \definecolor{ansi-green-intense}{HTML}{007427}
    \definecolor{ansi-yellow}{HTML}{DDB62B}
    \definecolor{ansi-yellow-intense}{HTML}{B27D12}
    \definecolor{ansi-blue}{HTML}{208FFB}
    \definecolor{ansi-blue-intense}{HTML}{0065CA}
    \definecolor{ansi-magenta}{HTML}{D160C4}
    \definecolor{ansi-magenta-intense}{HTML}{A03196}
    \definecolor{ansi-cyan}{HTML}{60C6C8}
    \definecolor{ansi-cyan-intense}{HTML}{258F8F}
    \definecolor{ansi-white}{HTML}{C5C1B4}
    \definecolor{ansi-white-intense}{HTML}{A1A6B2}

    % commands and environments needed by pandoc snippets
    % extracted from the output of `pandoc -s`
    \providecommand{\tightlist}{%
      \setlength{\itemsep}{0pt}\setlength{\parskip}{0pt}}
    \DefineVerbatimEnvironment{Highlighting}{Verbatim}{commandchars=\\\{\}}
    % Add ',fontsize=\small' for more characters per line
    \newenvironment{Shaded}{}{}
    \newcommand{\KeywordTok}[1]{\textcolor[rgb]{0.00,0.44,0.13}{\textbf{{#1}}}}
    \newcommand{\DataTypeTok}[1]{\textcolor[rgb]{0.56,0.13,0.00}{{#1}}}
    \newcommand{\DecValTok}[1]{\textcolor[rgb]{0.25,0.63,0.44}{{#1}}}
    \newcommand{\BaseNTok}[1]{\textcolor[rgb]{0.25,0.63,0.44}{{#1}}}
    \newcommand{\FloatTok}[1]{\textcolor[rgb]{0.25,0.63,0.44}{{#1}}}
    \newcommand{\CharTok}[1]{\textcolor[rgb]{0.25,0.44,0.63}{{#1}}}
    \newcommand{\StringTok}[1]{\textcolor[rgb]{0.25,0.44,0.63}{{#1}}}
    \newcommand{\CommentTok}[1]{\textcolor[rgb]{0.38,0.63,0.69}{\textit{{#1}}}}
    \newcommand{\OtherTok}[1]{\textcolor[rgb]{0.00,0.44,0.13}{{#1}}}
    \newcommand{\AlertTok}[1]{\textcolor[rgb]{1.00,0.00,0.00}{\textbf{{#1}}}}
    \newcommand{\FunctionTok}[1]{\textcolor[rgb]{0.02,0.16,0.49}{{#1}}}
    \newcommand{\RegionMarkerTok}[1]{{#1}}
    \newcommand{\ErrorTok}[1]{\textcolor[rgb]{1.00,0.00,0.00}{\textbf{{#1}}}}
    \newcommand{\NormalTok}[1]{{#1}}
    
    % Additional commands for more recent versions of Pandoc
    \newcommand{\ConstantTok}[1]{\textcolor[rgb]{0.53,0.00,0.00}{{#1}}}
    \newcommand{\SpecialCharTok}[1]{\textcolor[rgb]{0.25,0.44,0.63}{{#1}}}
    \newcommand{\VerbatimStringTok}[1]{\textcolor[rgb]{0.25,0.44,0.63}{{#1}}}
    \newcommand{\SpecialStringTok}[1]{\textcolor[rgb]{0.73,0.40,0.53}{{#1}}}
    \newcommand{\ImportTok}[1]{{#1}}
    \newcommand{\DocumentationTok}[1]{\textcolor[rgb]{0.73,0.13,0.13}{\textit{{#1}}}}
    \newcommand{\AnnotationTok}[1]{\textcolor[rgb]{0.38,0.63,0.69}{\textbf{\textit{{#1}}}}}
    \newcommand{\CommentVarTok}[1]{\textcolor[rgb]{0.38,0.63,0.69}{\textbf{\textit{{#1}}}}}
    \newcommand{\VariableTok}[1]{\textcolor[rgb]{0.10,0.09,0.49}{{#1}}}
    \newcommand{\ControlFlowTok}[1]{\textcolor[rgb]{0.00,0.44,0.13}{\textbf{{#1}}}}
    \newcommand{\OperatorTok}[1]{\textcolor[rgb]{0.40,0.40,0.40}{{#1}}}
    \newcommand{\BuiltInTok}[1]{{#1}}
    \newcommand{\ExtensionTok}[1]{{#1}}
    \newcommand{\PreprocessorTok}[1]{\textcolor[rgb]{0.74,0.48,0.00}{{#1}}}
    \newcommand{\AttributeTok}[1]{\textcolor[rgb]{0.49,0.56,0.16}{{#1}}}
    \newcommand{\InformationTok}[1]{\textcolor[rgb]{0.38,0.63,0.69}{\textbf{\textit{{#1}}}}}
    \newcommand{\WarningTok}[1]{\textcolor[rgb]{0.38,0.63,0.69}{\textbf{\textit{{#1}}}}}
    
    
    % Define a nice break command that doesn't care if a line doesn't already
    % exist.
    \def\br{\hspace*{\fill} \\* }
    % Math Jax compatability definitions
    \def\gt{>}
    \def\lt{<}
    % Document parameters
    \title{01-Titanic}
    
    
    

    % Pygments definitions
    
\makeatletter
\def\PY@reset{\let\PY@it=\relax \let\PY@bf=\relax%
    \let\PY@ul=\relax \let\PY@tc=\relax%
    \let\PY@bc=\relax \let\PY@ff=\relax}
\def\PY@tok#1{\csname PY@tok@#1\endcsname}
\def\PY@toks#1+{\ifx\relax#1\empty\else%
    \PY@tok{#1}\expandafter\PY@toks\fi}
\def\PY@do#1{\PY@bc{\PY@tc{\PY@ul{%
    \PY@it{\PY@bf{\PY@ff{#1}}}}}}}
\def\PY#1#2{\PY@reset\PY@toks#1+\relax+\PY@do{#2}}

\expandafter\def\csname PY@tok@w\endcsname{\def\PY@tc##1{\textcolor[rgb]{0.73,0.73,0.73}{##1}}}
\expandafter\def\csname PY@tok@c\endcsname{\let\PY@it=\textit\def\PY@tc##1{\textcolor[rgb]{0.25,0.50,0.50}{##1}}}
\expandafter\def\csname PY@tok@cp\endcsname{\def\PY@tc##1{\textcolor[rgb]{0.74,0.48,0.00}{##1}}}
\expandafter\def\csname PY@tok@k\endcsname{\let\PY@bf=\textbf\def\PY@tc##1{\textcolor[rgb]{0.00,0.50,0.00}{##1}}}
\expandafter\def\csname PY@tok@kp\endcsname{\def\PY@tc##1{\textcolor[rgb]{0.00,0.50,0.00}{##1}}}
\expandafter\def\csname PY@tok@kt\endcsname{\def\PY@tc##1{\textcolor[rgb]{0.69,0.00,0.25}{##1}}}
\expandafter\def\csname PY@tok@o\endcsname{\def\PY@tc##1{\textcolor[rgb]{0.40,0.40,0.40}{##1}}}
\expandafter\def\csname PY@tok@ow\endcsname{\let\PY@bf=\textbf\def\PY@tc##1{\textcolor[rgb]{0.67,0.13,1.00}{##1}}}
\expandafter\def\csname PY@tok@nb\endcsname{\def\PY@tc##1{\textcolor[rgb]{0.00,0.50,0.00}{##1}}}
\expandafter\def\csname PY@tok@nf\endcsname{\def\PY@tc##1{\textcolor[rgb]{0.00,0.00,1.00}{##1}}}
\expandafter\def\csname PY@tok@nc\endcsname{\let\PY@bf=\textbf\def\PY@tc##1{\textcolor[rgb]{0.00,0.00,1.00}{##1}}}
\expandafter\def\csname PY@tok@nn\endcsname{\let\PY@bf=\textbf\def\PY@tc##1{\textcolor[rgb]{0.00,0.00,1.00}{##1}}}
\expandafter\def\csname PY@tok@ne\endcsname{\let\PY@bf=\textbf\def\PY@tc##1{\textcolor[rgb]{0.82,0.25,0.23}{##1}}}
\expandafter\def\csname PY@tok@nv\endcsname{\def\PY@tc##1{\textcolor[rgb]{0.10,0.09,0.49}{##1}}}
\expandafter\def\csname PY@tok@no\endcsname{\def\PY@tc##1{\textcolor[rgb]{0.53,0.00,0.00}{##1}}}
\expandafter\def\csname PY@tok@nl\endcsname{\def\PY@tc##1{\textcolor[rgb]{0.63,0.63,0.00}{##1}}}
\expandafter\def\csname PY@tok@ni\endcsname{\let\PY@bf=\textbf\def\PY@tc##1{\textcolor[rgb]{0.60,0.60,0.60}{##1}}}
\expandafter\def\csname PY@tok@na\endcsname{\def\PY@tc##1{\textcolor[rgb]{0.49,0.56,0.16}{##1}}}
\expandafter\def\csname PY@tok@nt\endcsname{\let\PY@bf=\textbf\def\PY@tc##1{\textcolor[rgb]{0.00,0.50,0.00}{##1}}}
\expandafter\def\csname PY@tok@nd\endcsname{\def\PY@tc##1{\textcolor[rgb]{0.67,0.13,1.00}{##1}}}
\expandafter\def\csname PY@tok@s\endcsname{\def\PY@tc##1{\textcolor[rgb]{0.73,0.13,0.13}{##1}}}
\expandafter\def\csname PY@tok@sd\endcsname{\let\PY@it=\textit\def\PY@tc##1{\textcolor[rgb]{0.73,0.13,0.13}{##1}}}
\expandafter\def\csname PY@tok@si\endcsname{\let\PY@bf=\textbf\def\PY@tc##1{\textcolor[rgb]{0.73,0.40,0.53}{##1}}}
\expandafter\def\csname PY@tok@se\endcsname{\let\PY@bf=\textbf\def\PY@tc##1{\textcolor[rgb]{0.73,0.40,0.13}{##1}}}
\expandafter\def\csname PY@tok@sr\endcsname{\def\PY@tc##1{\textcolor[rgb]{0.73,0.40,0.53}{##1}}}
\expandafter\def\csname PY@tok@ss\endcsname{\def\PY@tc##1{\textcolor[rgb]{0.10,0.09,0.49}{##1}}}
\expandafter\def\csname PY@tok@sx\endcsname{\def\PY@tc##1{\textcolor[rgb]{0.00,0.50,0.00}{##1}}}
\expandafter\def\csname PY@tok@m\endcsname{\def\PY@tc##1{\textcolor[rgb]{0.40,0.40,0.40}{##1}}}
\expandafter\def\csname PY@tok@gh\endcsname{\let\PY@bf=\textbf\def\PY@tc##1{\textcolor[rgb]{0.00,0.00,0.50}{##1}}}
\expandafter\def\csname PY@tok@gu\endcsname{\let\PY@bf=\textbf\def\PY@tc##1{\textcolor[rgb]{0.50,0.00,0.50}{##1}}}
\expandafter\def\csname PY@tok@gd\endcsname{\def\PY@tc##1{\textcolor[rgb]{0.63,0.00,0.00}{##1}}}
\expandafter\def\csname PY@tok@gi\endcsname{\def\PY@tc##1{\textcolor[rgb]{0.00,0.63,0.00}{##1}}}
\expandafter\def\csname PY@tok@gr\endcsname{\def\PY@tc##1{\textcolor[rgb]{1.00,0.00,0.00}{##1}}}
\expandafter\def\csname PY@tok@ge\endcsname{\let\PY@it=\textit}
\expandafter\def\csname PY@tok@gs\endcsname{\let\PY@bf=\textbf}
\expandafter\def\csname PY@tok@gp\endcsname{\let\PY@bf=\textbf\def\PY@tc##1{\textcolor[rgb]{0.00,0.00,0.50}{##1}}}
\expandafter\def\csname PY@tok@go\endcsname{\def\PY@tc##1{\textcolor[rgb]{0.53,0.53,0.53}{##1}}}
\expandafter\def\csname PY@tok@gt\endcsname{\def\PY@tc##1{\textcolor[rgb]{0.00,0.27,0.87}{##1}}}
\expandafter\def\csname PY@tok@err\endcsname{\def\PY@bc##1{\setlength{\fboxsep}{0pt}\fcolorbox[rgb]{1.00,0.00,0.00}{1,1,1}{\strut ##1}}}
\expandafter\def\csname PY@tok@kc\endcsname{\let\PY@bf=\textbf\def\PY@tc##1{\textcolor[rgb]{0.00,0.50,0.00}{##1}}}
\expandafter\def\csname PY@tok@kd\endcsname{\let\PY@bf=\textbf\def\PY@tc##1{\textcolor[rgb]{0.00,0.50,0.00}{##1}}}
\expandafter\def\csname PY@tok@kn\endcsname{\let\PY@bf=\textbf\def\PY@tc##1{\textcolor[rgb]{0.00,0.50,0.00}{##1}}}
\expandafter\def\csname PY@tok@kr\endcsname{\let\PY@bf=\textbf\def\PY@tc##1{\textcolor[rgb]{0.00,0.50,0.00}{##1}}}
\expandafter\def\csname PY@tok@bp\endcsname{\def\PY@tc##1{\textcolor[rgb]{0.00,0.50,0.00}{##1}}}
\expandafter\def\csname PY@tok@fm\endcsname{\def\PY@tc##1{\textcolor[rgb]{0.00,0.00,1.00}{##1}}}
\expandafter\def\csname PY@tok@vc\endcsname{\def\PY@tc##1{\textcolor[rgb]{0.10,0.09,0.49}{##1}}}
\expandafter\def\csname PY@tok@vg\endcsname{\def\PY@tc##1{\textcolor[rgb]{0.10,0.09,0.49}{##1}}}
\expandafter\def\csname PY@tok@vi\endcsname{\def\PY@tc##1{\textcolor[rgb]{0.10,0.09,0.49}{##1}}}
\expandafter\def\csname PY@tok@vm\endcsname{\def\PY@tc##1{\textcolor[rgb]{0.10,0.09,0.49}{##1}}}
\expandafter\def\csname PY@tok@sa\endcsname{\def\PY@tc##1{\textcolor[rgb]{0.73,0.13,0.13}{##1}}}
\expandafter\def\csname PY@tok@sb\endcsname{\def\PY@tc##1{\textcolor[rgb]{0.73,0.13,0.13}{##1}}}
\expandafter\def\csname PY@tok@sc\endcsname{\def\PY@tc##1{\textcolor[rgb]{0.73,0.13,0.13}{##1}}}
\expandafter\def\csname PY@tok@dl\endcsname{\def\PY@tc##1{\textcolor[rgb]{0.73,0.13,0.13}{##1}}}
\expandafter\def\csname PY@tok@s2\endcsname{\def\PY@tc##1{\textcolor[rgb]{0.73,0.13,0.13}{##1}}}
\expandafter\def\csname PY@tok@sh\endcsname{\def\PY@tc##1{\textcolor[rgb]{0.73,0.13,0.13}{##1}}}
\expandafter\def\csname PY@tok@s1\endcsname{\def\PY@tc##1{\textcolor[rgb]{0.73,0.13,0.13}{##1}}}
\expandafter\def\csname PY@tok@mb\endcsname{\def\PY@tc##1{\textcolor[rgb]{0.40,0.40,0.40}{##1}}}
\expandafter\def\csname PY@tok@mf\endcsname{\def\PY@tc##1{\textcolor[rgb]{0.40,0.40,0.40}{##1}}}
\expandafter\def\csname PY@tok@mh\endcsname{\def\PY@tc##1{\textcolor[rgb]{0.40,0.40,0.40}{##1}}}
\expandafter\def\csname PY@tok@mi\endcsname{\def\PY@tc##1{\textcolor[rgb]{0.40,0.40,0.40}{##1}}}
\expandafter\def\csname PY@tok@il\endcsname{\def\PY@tc##1{\textcolor[rgb]{0.40,0.40,0.40}{##1}}}
\expandafter\def\csname PY@tok@mo\endcsname{\def\PY@tc##1{\textcolor[rgb]{0.40,0.40,0.40}{##1}}}
\expandafter\def\csname PY@tok@ch\endcsname{\let\PY@it=\textit\def\PY@tc##1{\textcolor[rgb]{0.25,0.50,0.50}{##1}}}
\expandafter\def\csname PY@tok@cm\endcsname{\let\PY@it=\textit\def\PY@tc##1{\textcolor[rgb]{0.25,0.50,0.50}{##1}}}
\expandafter\def\csname PY@tok@cpf\endcsname{\let\PY@it=\textit\def\PY@tc##1{\textcolor[rgb]{0.25,0.50,0.50}{##1}}}
\expandafter\def\csname PY@tok@c1\endcsname{\let\PY@it=\textit\def\PY@tc##1{\textcolor[rgb]{0.25,0.50,0.50}{##1}}}
\expandafter\def\csname PY@tok@cs\endcsname{\let\PY@it=\textit\def\PY@tc##1{\textcolor[rgb]{0.25,0.50,0.50}{##1}}}

\def\PYZbs{\char`\\}
\def\PYZus{\char`\_}
\def\PYZob{\char`\{}
\def\PYZcb{\char`\}}
\def\PYZca{\char`\^}
\def\PYZam{\char`\&}
\def\PYZlt{\char`\<}
\def\PYZgt{\char`\>}
\def\PYZsh{\char`\#}
\def\PYZpc{\char`\%}
\def\PYZdl{\char`\$}
\def\PYZhy{\char`\-}
\def\PYZsq{\char`\'}
\def\PYZdq{\char`\"}
\def\PYZti{\char`\~}
% for compatibility with earlier versions
\def\PYZat{@}
\def\PYZlb{[}
\def\PYZrb{]}
\makeatother


    % Exact colors from NB
    \definecolor{incolor}{rgb}{0.0, 0.0, 0.5}
    \definecolor{outcolor}{rgb}{0.545, 0.0, 0.0}



    
    % Prevent overflowing lines due to hard-to-break entities
    \sloppy 
    % Setup hyperref package
    \hypersetup{
      breaklinks=true,  % so long urls are correctly broken across lines
      colorlinks=true,
      urlcolor=urlcolor,
      linkcolor=linkcolor,
      citecolor=citecolor,
      }
    % Slightly bigger margins than the latex defaults
    
    \geometry{verbose,tmargin=1in,bmargin=1in,lmargin=1in,rmargin=1in}
    
    

    \begin{document}
    
    
    \maketitle
    
    

    
    \hypertarget{the-titanic-data-set}{%
\section{The Titanic Data Set}\label{the-titanic-data-set}}

Next, we will use one of the most famous data sets to begin to learn
about exploratory data analysis and visualization. First, we define the
task in terms of a well-defined problem statement.

    ** Domain** This is an introductory data set considered the ``hello
world'' of data science. It is an ongoing competition on Kaggle allowing
students of data science to prepare a model and make a submission to a
competition while they are still learning the subject.

\textbf{Problem} This is a binary classification problem in which the
challenge is to predict whether a passenger survived the sinking of the
Titanic given the demographic data of the passengers. Here, the task
\(T\) is a binary classification and the experience \(E\) is the list of
passengers and their survival outcome.

\textbf{Solution} To solve this problem, we will generate a vector of
integers using filtering and masking.

\textbf{Data} The following analysis shows:

\begin{itemize}
\item
  there are 891 rows and 10 useful variable columns in the dataset. One
  of these columns is the target \texttt{Survived}. An 11th and 12th
  column are a unique id for each passenger and the name of each
  passenger, respectively, and have no predictive power.
\item
  there are four integer value columns:

  \begin{itemize}
  \tightlist
  \item
    \texttt{Survived}
  \item
    \texttt{Pclass}
  \item
    \texttt{SibSp}
  \item
    \texttt{Parch}
  \end{itemize}
\item
  there are two numerical value columns:

  \begin{itemize}
  \tightlist
  \item
    \texttt{Age}
  \item
    \texttt{Fare}
  \end{itemize}
\item
  there are five factor columns:

  \begin{itemize}
  \tightlist
  \item
    \texttt{Sex}
  \item
    \texttt{Ticket}
  \item
    \texttt{Cabin}
  \item
    \texttt{Embarked}
  \end{itemize}
\item
  The following are the summary statistics of the data:

\begin{verbatim}
   Survived     Pclass      Sex         Age          SibSp      Parch           Fare         
 Min. :0.0000 Min. :1.000 female:314 Min. : 0.42 Min. :0.000 Min. :0.0000 Min. :  0.00 
 Mean :0.3838 Mean :2.309 male  :577 Mean :29.70 Mean :0.523 Mean :0.3816 Mean : 32.20 
 Max. :1.0000 Max. :3.000            Max. :80.00 Max. :8.000 Max. :6.0000 Max. :512.33 
                                     NA's   :177  
\end{verbatim}
\end{itemize}

\textbf{Benchmark} We will use a naive guess based on the most common
class as a benchmark. 61.6\% of passengers did not survive. We will
guess for our benchmark that there were no survivors.

\textbf{Metrics} As this is an beginning exercise, we will use the
accuracy.

    \hypertarget{measuring-accuracy}{%
\subsection{Measuring Accuracy}\label{measuring-accuracy}}

We have written two functions here to help us to measure the accuracy of
a prediction vector. The first function is called
\texttt{verify\_length}. It takes two vectors and compares their length
to make sure that they have the same length. This function is used in
the second function as a preliminary check. If a prediction vector does
not have the same length as a vector of actual values then there is a
deeper problem that must be dealt with.

The second function is the \texttt{accuracy} function. This function
takes two vectors: 1) a vector of actual values and 2) a vector of
predicted values and compares them. It assigns a value of true to each
value that the prediction gets correct. Finally, all of the true values
are counted and this is divided by the length of the vector of actual
values.

    \hypertarget{define-accuracy-metric}{%
\subparagraph{define accuracy metric}\label{define-accuracy-metric}}

    \begin{Verbatim}[commandchars=\\\{\}]
{\color{incolor}In [{\color{incolor}1}]:} verify\PYZus{}length \PY{o}{\PYZlt{}\PYZhy{}} \PY{k+kr}{function} \PY{p}{(}v1\PY{p}{,} v2 \PY{p}{)}\PY{p}{\PYZob{}}
            \PY{k+kr}{if} \PY{p}{(}\PY{k+kp}{length}\PY{p}{(}v1\PY{p}{)} \PY{o}{!=} \PY{k+kp}{length}\PY{p}{(}v2\PY{p}{)}\PY{p}{)} \PY{p}{\PYZob{}}
                \PY{k+kp}{stop}\PY{p}{(}\PY{l+s}{\PYZsq{}}\PY{l+s}{length of vectors do not match\PYZsq{}}\PY{p}{)} 
            \PY{p}{\PYZcb{}}
        \PY{p}{\PYZcb{}}
        
        accuracy \PY{o}{\PYZlt{}\PYZhy{}} \PY{k+kr}{function} \PY{p}{(}actual\PY{p}{,} predicted\PY{p}{)} \PY{p}{\PYZob{}}
            verify\PYZus{}length\PY{p}{(}actual\PY{p}{,} predicted\PY{p}{)}
            \PY{k+kr}{return}\PY{p}{(}\PY{k+kp}{sum}\PY{p}{(}actual \PY{o}{==} predicted\PY{p}{)}\PY{o}{/}\PY{k+kp}{length}\PY{p}{(}actual\PY{p}{)}\PY{p}{)}
        \PY{p}{\PYZcb{}}
\end{Verbatim}


    For example we might have the following vector of the actual values:

    \hypertarget{a-simple-vector-of-actual-values}{%
\subparagraph{a simple vector of actual
values}\label{a-simple-vector-of-actual-values}}

    \begin{Verbatim}[commandchars=\\\{\}]
{\color{incolor}In [{\color{incolor}2}]:} actual \PY{o}{=} \PY{k+kt}{c}\PY{p}{(}\PY{l+m}{1}\PY{p}{,}\PY{l+m}{1}\PY{p}{,}\PY{l+m}{0}\PY{p}{,}\PY{l+m}{0}\PY{p}{,}\PY{l+m}{1}\PY{p}{)}
\end{Verbatim}


    Our model might generate the following vector of predicted values:

    \hypertarget{a-simple-vector-of-predictions}{%
\subparagraph{a simple vector of
predictions}\label{a-simple-vector-of-predictions}}

    \begin{Verbatim}[commandchars=\\\{\}]
{\color{incolor}In [{\color{incolor}3}]:} predicted \PY{o}{=} \PY{k+kt}{c}\PY{p}{(}\PY{l+m}{1}\PY{p}{,}\PY{l+m}{1}\PY{p}{,}\PY{l+m}{1}\PY{p}{,}\PY{l+m}{0}\PY{p}{,}\PY{l+m}{0}\PY{p}{)}
\end{Verbatim}


    For this simple result, we can look at it and tell that the predictions
get 3 right and 2 wrong for an accuracy of 0.6.

    \hypertarget{assess-accuracy-of-predictions}{%
\subparagraph{assess accuracy of
predictions}\label{assess-accuracy-of-predictions}}

    \begin{Verbatim}[commandchars=\\\{\}]
{\color{incolor}In [{\color{incolor}4}]:} accuracy\PY{p}{(}actual\PY{p}{,} predicted\PY{p}{)}
\end{Verbatim}


    0.6

    
    \hypertarget{preliminary-analysis}{%
\subsection{Preliminary Analysis}\label{preliminary-analysis}}

We will start with some preliminary analysis on our data set.

    \hypertarget{load-the-dataset-using-r}{%
\subsubsection{Load the dataset using
R}\label{load-the-dataset-using-r}}

First, we love the data set using the R function \texttt{read.csv} and
assign it to the variable \texttt{titanic}. Note that the
\texttt{read.table} and \texttt{read.csv} in R are equivalent accept for
the default args. \texttt{read.table} defaults to separating on white
space. \texttt{read.csv} defaults to separating on commas.
\texttt{read.csv} also defaults to the argument \texttt{header=T}.

    \hypertarget{load-the-dataset-using-read.csv}{%
\subparagraph{\texorpdfstring{load the dataset using
\texttt{read.csv()}}{load the dataset using read.csv()}}\label{load-the-dataset-using-read.csv}}

    \begin{Verbatim}[commandchars=\\\{\}]
{\color{incolor}In [{\color{incolor}5}]:} titanic \PY{o}{\PYZlt{}\PYZhy{}} read.csv\PY{p}{(}\PY{l+s}{\PYZsq{}}\PY{l+s}{train.csv\PYZsq{}}\PY{p}{)}
\end{Verbatim}


    \begin{Verbatim}[commandchars=\\\{\}]
{\color{incolor}In [{\color{incolor}6}]:} \PY{k+kp}{stopifnot}\PY{p}{(}\PY{k+kp}{dim}\PY{p}{(}titanic\PY{p}{)} \PY{o}{==} \PY{k+kt}{c}\PY{p}{(}\PY{l+m}{891}\PY{p}{,}\PY{l+m}{12}\PY{p}{)}\PY{p}{)}
\end{Verbatim}


    We displayed the dimension \texttt{dim()} and the structure
\texttt{str()} of our datafrane. This is mostly done as a sanity check.
We should have some idea of what the dimension and structure of our data
is. By displaying these results immediately after loading the data, we
can verify that the data has been loaded as we expect.

    \hypertarget{display-the-dimension-of-the-data-set}{%
\subparagraph{display the dimension of the data
set}\label{display-the-dimension-of-the-data-set}}

    \begin{Verbatim}[commandchars=\\\{\}]
{\color{incolor}In [{\color{incolor}7}]:} \PY{k+kp}{dim}\PY{p}{(}titanic\PY{p}{)}
\end{Verbatim}


    \begin{enumerate*}
\item 891
\item 12
\end{enumerate*}


    
    \hypertarget{display-the-structure-of-the-dataframe}{%
\subparagraph{display the structure of the
dataframe}\label{display-the-structure-of-the-dataframe}}

    \begin{Verbatim}[commandchars=\\\{\}]
{\color{incolor}In [{\color{incolor}8}]:} str\PY{p}{(}titanic\PY{p}{)}
\end{Verbatim}


    \begin{Verbatim}[commandchars=\\\{\}]
'data.frame':	891 obs. of  12 variables:
 \$ PassengerId: int  1 2 3 4 5 6 7 8 9 10 {\ldots}
 \$ Survived   : int  0 1 1 1 0 0 0 0 1 1 {\ldots}
 \$ Pclass     : int  3 1 3 1 3 3 1 3 3 2 {\ldots}
 \$ Name       : Factor w/ 891 levels "Abbing, Mr. Anthony",..: 109 191 358 277 16 559 520 629 417 581 {\ldots}
 \$ Sex        : Factor w/ 2 levels "female","male": 2 1 1 1 2 2 2 2 1 1 {\ldots}
 \$ Age        : num  22 38 26 35 35 NA 54 2 27 14 {\ldots}
 \$ SibSp      : int  1 1 0 1 0 0 0 3 0 1 {\ldots}
 \$ Parch      : int  0 0 0 0 0 0 0 1 2 0 {\ldots}
 \$ Ticket     : Factor w/ 681 levels "110152","110413",..: 524 597 670 50 473 276 86 396 345 133 {\ldots}
 \$ Fare       : num  7.25 71.28 7.92 53.1 8.05 {\ldots}
 \$ Cabin      : Factor w/ 148 levels "","A10","A14",..: 1 83 1 57 1 1 131 1 1 1 {\ldots}
 \$ Embarked   : Factor w/ 4 levels "","C","Q","S": 4 2 4 4 4 3 4 4 4 2 {\ldots}

    \end{Verbatim}

    \hypertarget{the-r-structure-object}{%
\subsubsection{\texorpdfstring{The \texttt{R} Structure
Object}{The R Structure Object}}\label{the-r-structure-object}}

    I interpret the structure of our data frame in the following way. Each
row in the structure object, \texttt{str(titanic)} represents a column
in the data frame \texttt{titanic}. The value immediately following the
\texttt{\$} is the name of that column. The value immediately following
the \texttt{:} is the data type of that column. The values following the
datatype are the first few values of the data in the column itself.

    Note that R has made some default decisions about the structure of our
data. It has designated five columns as integer columns, five columns as
factor columns, and two columns as numerical problems. These may or may
not be accurate according to our own understanding of the data. This was
done by R, doing its best to intuit the structure of the data during the
read of the CSV file. For example, a reasonable case could be made that
the the \texttt{Survived} column should not be an integer, nor should
the \texttt{Pclass}.

    \hypertarget{categorical-features-in-r}{%
\paragraph{Categorical Features In R}\label{categorical-features-in-r}}

R stores categorical features using a special type of vector called a
\textbf{factor}. The data is stored as a vector of integers. The factor
has an additional attribute, however. It also has a vector of levels.
The integer stored as data are actually references to the vector of
names. We can think of the data stored in the Factor as a mapping to the
vector of levels.

    \hypertarget{display-that-class-of-the-titanicembarked-column}{%
\subparagraph{\texorpdfstring{display that class of the
\texttt{titanic\$embarked}
column}{display that class of the titanic\$embarked column}}\label{display-that-class-of-the-titanicembarked-column}}

    \begin{Verbatim}[commandchars=\\\{\}]
{\color{incolor}In [{\color{incolor}9}]:} \PY{k+kp}{class}\PY{p}{(}titanic\PY{o}{\PYZdl{}}Embarked\PY{p}{)}
\end{Verbatim}


    'factor'

    
    \hypertarget{display-that-levels-of-the-titanicembarked-column}{%
\subparagraph{\texorpdfstring{display that levels of the
\texttt{titanic\$embarked}
column}{display that levels of the titanic\$embarked column}}\label{display-that-levels-of-the-titanicembarked-column}}

    \begin{Verbatim}[commandchars=\\\{\}]
{\color{incolor}In [{\color{incolor}10}]:} \PY{k+kp}{levels}\PY{p}{(}titanic\PY{o}{\PYZdl{}}Embarked\PY{p}{)}
\end{Verbatim}


    \begin{enumerate*}
\item ''
\item 'C'
\item 'Q'
\item 'S'
\end{enumerate*}


    
    \hypertarget{display-that-first-few-values-of-the-titanicembarked-column}{%
\subparagraph{\texorpdfstring{display that first few values of the
\texttt{titanic\$embarked}
column}{display that first few values of the titanic\$embarked column}}\label{display-that-first-few-values-of-the-titanicembarked-column}}

    \begin{Verbatim}[commandchars=\\\{\}]
{\color{incolor}In [{\color{incolor}11}]:} titanic\PY{o}{\PYZdl{}}Embarked\PY{p}{[}\PY{l+m}{1}\PY{o}{:}\PY{l+m}{5}\PY{p}{]}
\end{Verbatim}


    \begin{enumerate*}
\item S
\item C
\item S
\item S
\item S
\end{enumerate*}


    
    \hypertarget{completely-unique-columns}{%
\subsubsection{Completely Unique
Columns}\label{completely-unique-columns}}

We can see from the structure of our data frame that it contains two
columns that are completely unique. We are attempting to use the
patterns in our data to make predictions about the survival of
passengers during the Titanic disaster. This is done by identifying
patterns in the data. If they column is completely unique there is no
pattern to be identified there. Each passenger has its own unique value
and there is really no immediate way to associate these unique values
with each other. For this reason we will simply remove the completely
unique columns. Prior to doing this, however, we should verify that they
are in fact completely.

    The two columns in question are \texttt{PassengerId} and \texttt{Name}.
We will use the following method to establish that they are both
completely unique:

\begin{enumerate}
\def\labelenumi{\arabic{enumi}.}
\tightlist
\item
  We will take a measure of the number of passengers in the data set
\item
  We will take a measure of the number of unique values in each of the
  columns in question
\item
  If the values match we will consider the column safe for removal
\end{enumerate}

    \hypertarget{store-the-number-of-passengers}{%
\subparagraph{store the number of
passengers}\label{store-the-number-of-passengers}}

    \begin{Verbatim}[commandchars=\\\{\}]
{\color{incolor}In [{\color{incolor}12}]:} number\PYZus{}of\PYZus{}passengers \PY{o}{=} \PY{k+kp}{length}\PY{p}{(}titanic\PY{o}{\PYZdl{}}PassengerId\PY{p}{)}
         number\PYZus{}of\PYZus{}passengers
\end{Verbatim}


    891

    
    \hypertarget{display-the-length-of-the-unique-values-in-titanicpassengerid-and-titanicname}{%
\subparagraph{\texorpdfstring{display the length of the unique values in
\texttt{titanic\$passengerid} and
\texttt{titanic\$name}}{display the length of the unique values in titanic\$passengerid and titanic\$name}}\label{display-the-length-of-the-unique-values-in-titanicpassengerid-and-titanicname}}

    \begin{Verbatim}[commandchars=\\\{\}]
{\color{incolor}In [{\color{incolor}13}]:} \PY{k+kp}{length}\PY{p}{(}\PY{k+kp}{unique}\PY{p}{(}titanic\PY{o}{\PYZdl{}}PassengerId\PY{p}{)}\PY{p}{)}\PY{p}{;} \PY{k+kp}{length}\PY{p}{(}\PY{k+kp}{unique}\PY{p}{(}titanic\PY{o}{\PYZdl{}}Name\PY{p}{)}\PY{p}{)}
\end{Verbatim}


    891

    
    891

    
    We note that the values do indeed match, therefore, it is safe to drop
both of these columns from our dataframe. This can be done by assigning
the \texttt{NULL} value to the named column. For example, we might do
the following on a generic data frame and column

\begin{verbatim}
dataframe$mycolumn = NULL
\end{verbatim}

    \hypertarget{drop-the-columns-with-completely-unique-values}{%
\subparagraph{drop the columns with completely unique
values}\label{drop-the-columns-with-completely-unique-values}}

    \begin{Verbatim}[commandchars=\\\{\}]
{\color{incolor}In [{\color{incolor}14}]:} titanic\PY{o}{\PYZdl{}}PassengerId \PY{o}{\PYZlt{}\PYZhy{}} \PY{k+kc}{NULL}
         titanic\PY{o}{\PYZdl{}}Name \PY{o}{\PYZlt{}\PYZhy{}} \PY{k+kc}{NULL}
\end{Verbatim}


    \begin{Verbatim}[commandchars=\\\{\}]
{\color{incolor}In [{\color{incolor}15}]:} \PY{k+kp}{stopifnot}\PY{p}{(}\PY{k+kp}{is.null}\PY{p}{(}titanic\PY{o}{\PYZdl{}}PassengerID\PY{p}{)}\PY{p}{)}
         \PY{k+kp}{stopifnot}\PY{p}{(}\PY{k+kp}{is.null}\PY{p}{(}titanic\PY{o}{\PYZdl{}}Name\PY{p}{)}\PY{p}{)}
\end{Verbatim}


    \hypertarget{summarize-the-data}{%
\subsubsection{Summarize The Data}\label{summarize-the-data}}

Finally, having dropped the features deemed not immediately useful, we
display the summary statistics of the dataframe using the
\texttt{summary()} function. This function shows the quartile values of
the data as well as mean and median for numerical features and the
counts to the best of its ability for the factors.

    \begin{Verbatim}[commandchars=\\\{\}]
{\color{incolor}In [{\color{incolor}16}]:} \PY{k+kp}{summary}\PY{p}{(}titanic\PY{p}{)}
\end{Verbatim}


    
    \begin{verbatim}
    Survived          Pclass          Sex           Age            SibSp      
 Min.   :0.0000   Min.   :1.000   female:314   Min.   : 0.42   Min.   :0.000  
 1st Qu.:0.0000   1st Qu.:2.000   male  :577   1st Qu.:20.12   1st Qu.:0.000  
 Median :0.0000   Median :3.000                Median :28.00   Median :0.000  
 Mean   :0.3838   Mean   :2.309                Mean   :29.70   Mean   :0.523  
 3rd Qu.:1.0000   3rd Qu.:3.000                3rd Qu.:38.00   3rd Qu.:1.000  
 Max.   :1.0000   Max.   :3.000                Max.   :80.00   Max.   :8.000  
                                               NA's   :177                    
     Parch             Ticket         Fare                Cabin     Embarked
 Min.   :0.0000   1601    :  7   Min.   :  0.00              :687    :  2   
 1st Qu.:0.0000   347082  :  7   1st Qu.:  7.91   B96 B98    :  4   C:168   
 Median :0.0000   CA. 2343:  7   Median : 14.45   C23 C25 C27:  4   Q: 77   
 Mean   :0.3816   3101295 :  6   Mean   : 32.20   G6         :  4   S:644   
 3rd Qu.:0.0000   347088  :  6   3rd Qu.: 31.00   C22 C26    :  3           
 Max.   :6.0000   CA 2144 :  6   Max.   :512.33   D          :  3           
                  (Other) :852                    (Other)    :186           
    \end{verbatim}

    
    \hypertarget{preparing-a-benchmark-model}{%
\subsection{Preparing A Benchmark
Model}\label{preparing-a-benchmark-model}}

Having performed a preliminary analysis of the data, we move onto
preparing a benchmark model. First, we will do some analysis of the
target column. Based upon this analysis we will think about what the
best model for a benchmark might be.

We will make use of the R \texttt{table()} function to study the target
column. This function builds a contingency table of the counts
combinations of factor levels. Of course if only a Single column is
passed to the function, it will just return a simple count.

    \hypertarget{display-a-contingency-table-of-titanicsurvived}{%
\subparagraph{\texorpdfstring{display a contingency table of
\texttt{titanic\$survived}}{display a contingency table of titanic\$survived}}\label{display-a-contingency-table-of-titanicsurvived}}

    \begin{Verbatim}[commandchars=\\\{\}]
{\color{incolor}In [{\color{incolor}17}]:} \PY{k+kp}{table}\PY{p}{(}titanic\PY{o}{\PYZdl{}}Survived\PY{p}{)}
\end{Verbatim}


    
    \begin{verbatim}

  0   1 
549 342 
    \end{verbatim}

    
    From the result returns, we can see that the survival status is stored
As either is 0, corresponding to did not survive, or a 1 corresponding
to survived. We can use the helper function \texttt{prop.table()} to
express the results a contingency table as fractions. Here, we can see
that \(0.\bar{61}\) Of the passengers did not survive. One thing we
should immediately take note of is that our target column is not evenly
distributed. An \textbf{evenly distributed} target column would have the
exact same number of each possible outcome. As we grow in our data
science practice we will learn more about dealing with an evenly
distributed target. For now it is sufficient to simply take note of this
fact.

    \hypertarget{display-a-proportion-table-of-titanicsurvived}{%
\subparagraph{\texorpdfstring{display a proportion table of
\texttt{titanic\$survived}}{display a proportion table of titanic\$survived}}\label{display-a-proportion-table-of-titanicsurvived}}

    \begin{Verbatim}[commandchars=\\\{\}]
{\color{incolor}In [{\color{incolor}18}]:} \PY{k+kp}{prop.table}\PY{p}{(}\PY{k+kp}{table}\PY{p}{(}titanic\PY{o}{\PYZdl{}}Survived\PY{p}{)}\PY{p}{)}
\end{Verbatim}


    
    \begin{verbatim}

        0         1 
0.6161616 0.3838384 
    \end{verbatim}

    
    Below, we use a histogram to show once more that the target is not
evenly distributed. By default, the \texttt{hist()} function simply
shows the counts for each measured value.

    \hypertarget{display-a-histogram-of-titanicsurvived}{%
\subparagraph{\texorpdfstring{display a histogram of
\texttt{titanic\$Survived}}{display a histogram of titanic\$Survived}}\label{display-a-histogram-of-titanicsurvived}}

    \begin{Verbatim}[commandchars=\\\{\}]
{\color{incolor}In [{\color{incolor}19}]:} \PY{k+kn}{library}\PY{p}{(}repr\PY{p}{)}
         \PY{k+kp}{options}\PY{p}{(}repr.plot.width\PY{o}{=}\PY{l+m}{10}\PY{p}{,} repr.plot.height\PY{o}{=}\PY{l+m}{4}\PY{p}{)}
         
         hist\PY{p}{(}titanic\PY{o}{\PYZdl{}}Survived\PY{p}{)}
\end{Verbatim}


    \begin{center}
    \adjustimage{max size={0.9\linewidth}{0.9\paperheight}}{output_54_0.png}
    \end{center}
    { \hspace*{\fill} \\}
    
    \hypertarget{a-nauxefve-guess}{%
\subsubsection{A Naïve Guess}\label{a-nauxefve-guess}}

We will use a naive guess based on the most common class as a benchmark.
61.6\% of passengers did not survive. We will guess for our benchmark
that there were no survivors. Note that we have done very little work
and already have a better then 50-50 chance I've getting a correct
answer simply by guessing that no one survived. This is one
consideration for having an unevenly distributed target. Simply
measuring accuracy may not give us a realistic sense of how well our
model is doing. This is one reason why preparing a benchmark is so
important. Had we not prepared at benchmark we might think that a 55\%
accuracy is deceny because it's better than the simple 50-50. This
benchmark gives us a sense of what we need to do better than in order to
prepare a model that adds value to the situation.

    \hypertarget{create-a-vector-called-no_survivors-that-is-a-list-of-predictions-that-no-one-survived.}{%
\subparagraph{\texorpdfstring{Create a vector called
\texttt{no\_survivors} that is a list of predictions that no one
survived.}{Create a vector called no\_survivors that is a list of predictions that no one survived.}}\label{create-a-vector-called-no_survivors-that-is-a-list-of-predictions-that-no-one-survived.}}

To create such a vector using R, we will use the replicate
\texttt{rep()} function. This function takes a value and replicates it a
given number of times.

    \begin{Verbatim}[commandchars=\\\{\}]
{\color{incolor}In [{\color{incolor}20}]:} no\PYZus{}survivors \PY{o}{\PYZlt{}\PYZhy{}} \PY{k+kp}{rep}\PY{p}{(}\PY{l+m}{0}\PY{p}{,} number\PYZus{}of\PYZus{}passengers\PY{p}{)}
\end{Verbatim}


    \begin{Verbatim}[commandchars=\\\{\}]
{\color{incolor}In [{\color{incolor}21}]:} \PY{c+c1}{\PYZsh{} HIDDEN TEST}
\end{Verbatim}


    Once we have prepared this naïve guess, we can use the \texttt{accuracy}
function we defined earlier to assess our benchmark as a vector of
predictions.

    \hypertarget{accuracy-of-our-nauxefve-prediction}{%
\subparagraph{accuracy of our naïve
prediction}\label{accuracy-of-our-nauxefve-prediction}}

    \begin{Verbatim}[commandchars=\\\{\}]
{\color{incolor}In [{\color{incolor}22}]:} accuracy\PY{p}{(}titanic\PY{o}{\PYZdl{}}Survived\PY{p}{,} no\PYZus{}survivors\PY{p}{)}
\end{Verbatim}


    0.616161616161616

    
    As expected, we achieve an accuracy of \(0.\bar{61}\).

    \hypertarget{a-vectorized-solution-to-fizzbuzz}{%
\subsection{\texorpdfstring{A Vectorized Solution To
\texttt{fizzbuzz}}{A Vectorized Solution To fizzbuzz}}\label{a-vectorized-solution-to-fizzbuzz}}

\texttt{fizzbuzz} is a canonical ``coding interview'' problem. You might
want to read this humorous take by Joel Grus who attempts to use tensor
for to solve the problem:
http://joelgrus.com/2016/05/23/fizz-buzz-in-tensorflow/. The challenge
is to iterate over the numbers from 1 to 100, printing ``fizz'' if the
number is divisible by 3, ``buzz'' if the number is divisible by 5,
``fizzbuzz'' if the number is divisible by 15, and the number itself
otherwise. Typically this problem is solved using for-loops and if-else
statements and is used as a basic assessment of programming ability.
Such a solution might look like this

    \hypertarget{a-first-attempt-at-fizzbuzz}{%
\subparagraph{\texorpdfstring{a first attempt at
\texttt{fizzbuzz}}{a first attempt at fizzbuzz}}\label{a-first-attempt-at-fizzbuzz}}

    \begin{Verbatim}[commandchars=\\\{\}]
{\color{incolor}In [{\color{incolor}23}]:} fizzbuzz \PY{o}{=} \PY{k+kr}{function} \PY{p}{(}n\PY{p}{)} \PY{p}{\PYZob{}}
             \PY{k+kr}{for} \PY{p}{(}i \PY{k+kr}{in} \PY{l+m}{1}\PY{o}{:}n\PY{p}{)} \PY{p}{\PYZob{}}
                 \PY{k+kr}{if} \PY{p}{(}i \PY{o}{\PYZpc{}\PYZpc{}} \PY{l+m}{15} \PY{o}{==} \PY{l+m}{0}\PY{p}{)} \PY{k+kp}{print}\PY{p}{(}\PY{l+s}{\PYZdq{}}\PY{l+s}{fizzbuzz\PYZdq{}}\PY{p}{)}
                 \PY{k+kr}{else} \PY{k+kr}{if} \PY{p}{(}i \PY{o}{\PYZpc{}\PYZpc{}} \PY{l+m}{3} \PY{o}{==} \PY{l+m}{0}\PY{p}{)} \PY{k+kp}{print}\PY{p}{(}\PY{l+s}{\PYZdq{}}\PY{l+s}{fizz\PYZdq{}}\PY{p}{)}
                 \PY{k+kr}{else} \PY{k+kr}{if} \PY{p}{(}i \PY{o}{\PYZpc{}\PYZpc{}} \PY{l+m}{5} \PY{o}{==} \PY{l+m}{0}\PY{p}{)} \PY{k+kp}{print}\PY{p}{(}\PY{l+s}{\PYZdq{}}\PY{l+s}{buzz\PYZdq{}}\PY{p}{)}
                 \PY{k+kr}{else} \PY{k+kp}{print}\PY{p}{(}i\PY{p}{)}
             \PY{p}{\PYZcb{}}
         \PY{p}{\PYZcb{}}
         fizzbuzz\PY{p}{(}\PY{l+m}{15}\PY{p}{)}
\end{Verbatim}


    \begin{Verbatim}[commandchars=\\\{\}]
[1] 1
[1] 2
[1] "fizz"
[1] 4
[1] "buzz"
[1] "fizz"
[1] 7
[1] 8
[1] "fizz"
[1] "buzz"
[1] 11
[1] "fizz"
[1] 13
[1] 14
[1] "fizzbuzz"

    \end{Verbatim}

    It may be a bit much to come up with a solution to this problem using
tensorflow. It is, however, very useful to think about solving this
problem using masks and filters. Suppose we begin with a simple solution
vector as follows

    \hypertarget{start-the-solution-vector}{%
\subparagraph{\texorpdfstring{start the \texttt{solution}
vector}{start the solution vector}}\label{start-the-solution-vector}}

    \begin{Verbatim}[commandchars=\\\{\}]
{\color{incolor}In [{\color{incolor}24}]:} solution \PY{o}{=} \PY{l+m}{1}\PY{o}{:}\PY{l+m}{15}
         solution
\end{Verbatim}


    \begin{enumerate*}
\item 1
\item 2
\item 3
\item 4
\item 5
\item 6
\item 7
\item 8
\item 9
\item 10
\item 11
\item 12
\item 13
\item 14
\item 15
\end{enumerate*}


    
    The challenge is to replace the values we don't need with the correct
strings. Sure we can iterate over this list check the value to see if
it's divisible by three or five but using a vectorized solution we can
do it all at once.

    The Steps to doing this are as follows:

\begin{enumerate}
\def\labelenumi{\arabic{enumi}.}
\tightlist
\item
  Create a mask for a certain condition we might wish to check
\item
  Use that mask to restrict the values of the original \texttt{solution}
  we are looking at
\item
  Replace to values of the restricted vector with the appropriate string
\end{enumerate}

    First, we create a mask called \texttt{mod15\_mask}. Note, that when we
display it there is only a single \texttt{TRUE} value, in the position
where the value is divisible by 15 (and in this case is actually 15).

    \hypertarget{create-the-mod-15-mask}{%
\subparagraph{create the mod 15 mask}\label{create-the-mod-15-mask}}

    \begin{Verbatim}[commandchars=\\\{\}]
{\color{incolor}In [{\color{incolor}25}]:} mod15\PYZus{}mask \PY{o}{=} \PY{p}{(}solution \PY{o}{\PYZpc{}\PYZpc{}} \PY{l+m}{15} \PY{o}{==} \PY{l+m}{0}\PY{p}{)}
         mod15\PYZus{}mask
\end{Verbatim}


    \begin{enumerate*}
\item FALSE
\item FALSE
\item FALSE
\item FALSE
\item FALSE
\item FALSE
\item FALSE
\item FALSE
\item FALSE
\item FALSE
\item FALSE
\item FALSE
\item FALSE
\item FALSE
\item TRUE
\end{enumerate*}


    
    Next, we filter the \texttt{solution} using the \texttt{mod15\_mask}.

    \hypertarget{filter-solution-using-the-mind-15-mask}{%
\subparagraph{\texorpdfstring{filter \texttt{solution} using the mind 15
mask}{filter solution using the mind 15 mask}}\label{filter-solution-using-the-mind-15-mask}}

    \begin{Verbatim}[commandchars=\\\{\}]
{\color{incolor}In [{\color{incolor}26}]:} solution\PY{p}{[}mod15\PYZus{}mask\PY{p}{]}
\end{Verbatim}


    15

    
    Finally, we assign the filtered values the string \texttt{"fizzbuzz"}

    \hypertarget{assign-valued-to-the-filtered-solution-vector}{%
\subparagraph{\texorpdfstring{assign valued to the filtered
\texttt{solution}
vector}{assign valued to the filtered solution vector}}\label{assign-valued-to-the-filtered-solution-vector}}

    \begin{Verbatim}[commandchars=\\\{\}]
{\color{incolor}In [{\color{incolor}27}]:} solution\PY{p}{[}mod15\PYZus{}mask\PY{p}{]} \PY{o}{=} \PY{l+s}{\PYZdq{}}\PY{l+s}{fizzbuzz\PYZdq{}}
\end{Verbatim}


    Let's have a look at the current value of our solution.

    \begin{Verbatim}[commandchars=\\\{\}]
{\color{incolor}In [{\color{incolor}28}]:} solution
\end{Verbatim}


    \begin{enumerate*}
\item '1'
\item '2'
\item '3'
\item '4'
\item '5'
\item '6'
\item '7'
\item '8'
\item '9'
\item '10'
\item '11'
\item '12'
\item '13'
\item '14'
\item 'fizzbuzz'
\end{enumerate*}


    
    We can repeat this technique to build an entire solution to the problem.

    \hypertarget{a-vectorized-fizzbuzz}{%
\subparagraph{\texorpdfstring{a vectorized
\texttt{fizzbuzz}}{a vectorized fizzbuzz}}\label{a-vectorized-fizzbuzz}}

    \begin{Verbatim}[commandchars=\\\{\}]
{\color{incolor}In [{\color{incolor}29}]:} fizzbuzz \PY{o}{=} \PY{k+kr}{function} \PY{p}{(}n\PY{p}{)} \PY{p}{\PYZob{}}
             solution \PY{o}{=} \PY{l+m}{1}\PY{o}{:}n
             mod3\PYZus{}mask \PY{o}{=} \PY{p}{(}solution \PY{o}{\PYZpc{}\PYZpc{}} \PY{l+m}{3} \PY{o}{==} \PY{l+m}{0}\PY{p}{)}
             mod5\PYZus{}mask \PY{o}{=} \PY{p}{(}solution \PY{o}{\PYZpc{}\PYZpc{}} \PY{l+m}{5} \PY{o}{==} \PY{l+m}{0}\PY{p}{)}
             mod15\PYZus{}mask \PY{o}{=} \PY{p}{(}solution \PY{o}{\PYZpc{}\PYZpc{}} \PY{l+m}{15} \PY{o}{==} \PY{l+m}{0}\PY{p}{)}
             
             solution\PY{p}{[}mod3\PYZus{}mask\PY{p}{]} \PY{o}{=} \PY{l+s}{\PYZdq{}}\PY{l+s}{fizz\PYZdq{}}
             solution\PY{p}{[}mod5\PYZus{}mask\PY{p}{]} \PY{o}{=} \PY{l+s}{\PYZdq{}}\PY{l+s}{buzz\PYZdq{}}
             solution\PY{p}{[}mod15\PYZus{}mask\PY{p}{]} \PY{o}{=} \PY{l+s}{\PYZdq{}}\PY{l+s}{fizzbuzz\PYZdq{}}
             
             \PY{k+kp}{cat}\PY{p}{(}solution\PY{p}{,}sep\PY{o}{=}\PY{l+s}{\PYZdq{}}\PY{l+s}{\PYZbs{}n\PYZdq{}}\PY{p}{)}
         \PY{p}{\PYZcb{}}
         
         fizzbuzz\PY{p}{(}\PY{l+m}{15}\PY{p}{)}
\end{Verbatim}


    \begin{Verbatim}[commandchars=\\\{\}]
1
2
fizz
4
buzz
fizz
7
8
fizz
buzz
11
fizz
13
14
fizzbuzz

    \end{Verbatim}

    In terms of the why of doing a vectorized approach, there are tremendous
speed gains to be had implementing your algorithms using vectors rather
than loops. To read more about this, have a look at this blog post:
http://www.noamross.net/blog/2014/4/16/vectorization-in-r--why.html

    \hypertarget{incremental-model-improvement-with-filters-and-masks}{%
\subsection{Incremental Model Improvement With Filters And
Masks}\label{incremental-model-improvement-with-filters-and-masks}}

    And now begins the work of data scientist. We have established a
benchmark model. We should now begin to refine upon this model seeking
to continually improve the benchmark performance that we have. We can do
this by using exploratory data analysis to study the features,
especially as they relate to the target. If we find a feature that we
believe exhibits some pattern of correspondence to our target we can use
this to refine our model.

    For this project, we are going to think of our model as simply the
values stored in a vector of predictions. For example, we already have
one model, a model called \texttt{no\_survivors}, which is simply a
vector of zeros. To improve upon this model we will use a mask to reduce
the number of values we are looking at and then replace these values
with a 1.

    \hypertarget{randomized-model-improvement}{%
\subsubsection{Randomized Model
Improvement}\label{randomized-model-improvement}}

What if we try to improve our model by simply randomly replacing zeros
with one? We can do this using the \texttt{sample()} function

    \hypertarget{create-a-random-mask}{%
\subparagraph{create a random mask}\label{create-a-random-mask}}

    \begin{Verbatim}[commandchars=\\\{\}]
{\color{incolor}In [{\color{incolor}30}]:} random\PYZus{}mask \PY{o}{=} \PY{k+kp}{sample}\PY{p}{(}\PY{k+kt}{c}\PY{p}{(}\PY{k+kc}{TRUE}\PY{p}{,}\PY{k+kc}{FALSE}\PY{p}{)}\PY{p}{,} number\PYZus{}of\PYZus{}passengers\PY{p}{,} replace \PY{o}{=} \PY{k+kc}{TRUE}\PY{p}{)}
         random\PYZus{}mask\PY{p}{[}\PY{l+m}{1}\PY{o}{:}\PY{l+m}{10}\PY{p}{]}
\end{Verbatim}


    \begin{enumerate*}
\item FALSE
\item FALSE
\item FALSE
\item TRUE
\item TRUE
\item TRUE
\item FALSE
\item FALSE
\item FALSE
\item TRUE
\end{enumerate*}


    
    \hypertarget{duplicate-and-filter-to-create-random-model}{%
\subparagraph{duplicate and filter to create random
model}\label{duplicate-and-filter-to-create-random-model}}

    \begin{Verbatim}[commandchars=\\\{\}]
{\color{incolor}In [{\color{incolor}31}]:} random\PYZus{}model \PY{o}{=} \PY{k+kp}{rep}\PY{p}{(}no\PYZus{}survivors\PY{p}{)}
         random\PYZus{}model\PY{p}{[}random\PYZus{}mask\PY{p}{]} \PY{o}{=} \PY{l+m}{1}
\end{Verbatim}


    \hypertarget{assess-accuracy-of-random-model}{%
\subparagraph{assess accuracy of random
model}\label{assess-accuracy-of-random-model}}

    \begin{Verbatim}[commandchars=\\\{\}]
{\color{incolor}In [{\color{incolor}32}]:} accuracy\PY{p}{(}titanic\PY{o}{\PYZdl{}}Survived\PY{p}{,} random\PYZus{}model\PY{p}{)}
\end{Verbatim}


    0.496071829405163

    
    As suspected, simply guessing is not better than guessing all zeros. It
looks like we might actually justify our exorbitant salaries after all.

    \hypertarget{use-proportion-tables-to-look-at-survival-by-feature}{%
\subsubsection{Use Proportion Tables To Look At Survival By
Feature}\label{use-proportion-tables-to-look-at-survival-by-feature}}

    Previously, we use a proportion table to look at a single feature,
\texttt{Survived}. Next, We will use a proportion table to look at how
two features interact with each other. Let's look at the structure of
the dataframe again to remind ourselves which features we have available
to us.

    \hypertarget{display-the-structure-of-the-dataframe}{%
\subparagraph{display the structure of the
dataframe}\label{display-the-structure-of-the-dataframe}}

    \begin{Verbatim}[commandchars=\\\{\}]
{\color{incolor}In [{\color{incolor}33}]:} str\PY{p}{(}titanic\PY{p}{)}
\end{Verbatim}


    \begin{Verbatim}[commandchars=\\\{\}]
'data.frame':	891 obs. of  10 variables:
 \$ Survived: int  0 1 1 1 0 0 0 0 1 1 {\ldots}
 \$ Pclass  : int  3 1 3 1 3 3 1 3 3 2 {\ldots}
 \$ Sex     : Factor w/ 2 levels "female","male": 2 1 1 1 2 2 2 2 1 1 {\ldots}
 \$ Age     : num  22 38 26 35 35 NA 54 2 27 14 {\ldots}
 \$ SibSp   : int  1 1 0 1 0 0 0 3 0 1 {\ldots}
 \$ Parch   : int  0 0 0 0 0 0 0 1 2 0 {\ldots}
 \$ Ticket  : Factor w/ 681 levels "110152","110413",..: 524 597 670 50 473 276 86 396 345 133 {\ldots}
 \$ Fare    : num  7.25 71.28 7.92 53.1 8.05 {\ldots}
 \$ Cabin   : Factor w/ 148 levels "","A10","A14",..: 1 83 1 57 1 1 131 1 1 1 {\ldots}
 \$ Embarked: Factor w/ 4 levels "","C","Q","S": 4 2 4 4 4 3 4 4 4 2 {\ldots}

    \end{Verbatim}

    First, we look at the proportions of \texttt{Pclass} and
\texttt{Survived}. There are three different ways we can look at a
proportion table.

\begin{enumerate}
\def\labelenumi{\arabic{enumi}.}
\tightlist
\item
  The values of each combination as a proportion of the whole
\item
  The values in each row as a proportion of that row
\item
  The values in each column as a proportion of that column
\end{enumerate}

    \hypertarget{whole-proportions-of-pclass-versus-survived}{%
\subparagraph{\texorpdfstring{whole proportions of \texttt{Pclass}
versus
\texttt{Survived}}{whole proportions of Pclass versus Survived}}\label{whole-proportions-of-pclass-versus-survived}}

    \begin{Verbatim}[commandchars=\\\{\}]
{\color{incolor}In [{\color{incolor}34}]:} \PY{k+kp}{prop.table}\PY{p}{(}\PY{k+kp}{table}\PY{p}{(}titanic\PY{o}{\PYZdl{}}Pclass\PY{p}{,} titanic\PY{o}{\PYZdl{}}Survived\PY{p}{)}\PY{p}{)}
\end{Verbatim}


    
    \begin{verbatim}
   
             0          1
  1 0.08978676 0.15263749
  2 0.10886644 0.09764310
  3 0.41750842 0.13355780
    \end{verbatim}

    
    \hypertarget{proportions-of-pclass-versus-survived-by-row}{%
\subparagraph{\texorpdfstring{proportions of \texttt{Pclass} versus
\texttt{Survived} by
row}{proportions of Pclass versus Survived by row}}\label{proportions-of-pclass-versus-survived-by-row}}

    \begin{Verbatim}[commandchars=\\\{\}]
{\color{incolor}In [{\color{incolor}35}]:} \PY{k+kp}{prop.table}\PY{p}{(}\PY{k+kp}{table}\PY{p}{(}titanic\PY{o}{\PYZdl{}}Pclass\PY{p}{,} titanic\PY{o}{\PYZdl{}}Survived\PY{p}{)}\PY{p}{,} \PY{l+m}{1}\PY{p}{)}
\end{Verbatim}


    
    \begin{verbatim}
   
            0         1
  1 0.3703704 0.6296296
  2 0.5271739 0.4728261
  3 0.7576375 0.2423625
    \end{verbatim}

    
    \hypertarget{proportions-of-pclass-versus-survived-by-column}{%
\subparagraph{\texorpdfstring{proportions of \texttt{Pclass} versus
\texttt{Survived} by
column}{proportions of Pclass versus Survived by column}}\label{proportions-of-pclass-versus-survived-by-column}}

    \begin{Verbatim}[commandchars=\\\{\}]
{\color{incolor}In [{\color{incolor}36}]:} \PY{k+kp}{prop.table}\PY{p}{(}\PY{k+kp}{table}\PY{p}{(}titanic\PY{o}{\PYZdl{}}Pclass\PY{p}{,} titanic\PY{o}{\PYZdl{}}Survived\PY{p}{)}\PY{p}{,} \PY{l+m}{2}\PY{p}{)}
\end{Verbatim}


    
    \begin{verbatim}
   
            0         1
  1 0.1457195 0.3976608
  2 0.1766849 0.2543860
  3 0.6775956 0.3479532
    \end{verbatim}

    
    \hypertarget{whole-proportions-of-sex-versus-survived}{%
\subparagraph{\texorpdfstring{whole proportions of \texttt{Sex} versus
\texttt{Survived}}{whole proportions of Sex versus Survived}}\label{whole-proportions-of-sex-versus-survived}}

    \begin{Verbatim}[commandchars=\\\{\}]
{\color{incolor}In [{\color{incolor}37}]:} \PY{k+kp}{prop.table}\PY{p}{(}\PY{k+kp}{table}\PY{p}{(}titanic\PY{o}{\PYZdl{}}Sex\PY{p}{,} titanic\PY{o}{\PYZdl{}}Survived\PY{p}{)}\PY{p}{)}
\end{Verbatim}


    
    \begin{verbatim}
        
                  0          1
  female 0.09090909 0.26150393
  male   0.52525253 0.12233446
    \end{verbatim}

    
    \hypertarget{proportions-of-sex-versus-survived-by-row}{%
\subparagraph{\texorpdfstring{proportions of \texttt{Sex} versus
\texttt{Survived} by
row}{proportions of Sex versus Survived by row}}\label{proportions-of-sex-versus-survived-by-row}}

    \begin{Verbatim}[commandchars=\\\{\}]
{\color{incolor}In [{\color{incolor}38}]:} \PY{k+kp}{prop.table}\PY{p}{(}\PY{k+kp}{table}\PY{p}{(}titanic\PY{o}{\PYZdl{}}Sex\PY{p}{,} titanic\PY{o}{\PYZdl{}}Survived\PY{p}{)}\PY{p}{,} \PY{l+m}{1}\PY{p}{)}
\end{Verbatim}


    
    \begin{verbatim}
        
                 0         1
  female 0.2579618 0.7420382
  male   0.8110919 0.1889081
    \end{verbatim}

    
    \hypertarget{proportions-of-sex-versus-survived-by-column}{%
\subparagraph{\texorpdfstring{proportions of \texttt{Sex} versus
\texttt{Survived} by
column}{proportions of Sex versus Survived by column}}\label{proportions-of-sex-versus-survived-by-column}}

    \begin{Verbatim}[commandchars=\\\{\}]
{\color{incolor}In [{\color{incolor}39}]:} \PY{k+kp}{prop.table}\PY{p}{(}\PY{k+kp}{table}\PY{p}{(}titanic\PY{o}{\PYZdl{}}Sex\PY{p}{,} titanic\PY{o}{\PYZdl{}}Survived\PY{p}{)}\PY{p}{,} \PY{l+m}{2}\PY{p}{)}
\end{Verbatim}


    
    \begin{verbatim}
        
                 0         1
  female 0.1475410 0.6812865
  male   0.8524590 0.3187135
    \end{verbatim}

    
    \hypertarget{analyze-proportion-tables}{%
\paragraph{Analyze Proportion Tables}\label{analyze-proportion-tables}}

Using the results obtained about prepare an analysis of how these two
features can be used to predict whether or not someone survived the
sinking of the Titanic.

    Using this result, we can perform a prediction by placing predictions
based on the proportions in the table. For example, the prediction
should have 14.75410\% of the sample of women who didn't survive.

    \hypertarget{targeted-model-improvement}{%
\subsubsection{Targeted Model
Improvement}\label{targeted-model-improvement}}

We saw that randomly selecting values to be replaced by one did not
improve our model. What if we use some more intelligent way to select
values that should be replaced by a one in our vector of predictions? We
just looked at two features and identified some patterns that showed it
would be more likely to have survived the sinking of the ship. Based
upon this work we might decide that it would be a better model to
replace the prediction for all female passengers with a 1. We can do
that using masks and filters.

    \hypertarget{create-a-mask-of-just-women}{%
\subparagraph{create a mask of just
women}\label{create-a-mask-of-just-women}}

    \begin{Verbatim}[commandchars=\\\{\}]
{\color{incolor}In [{\color{incolor}40}]:} women\PYZus{}mask \PY{o}{=} titanic\PY{o}{\PYZdl{}}Sex \PY{o}{==} \PY{l+s}{\PYZsq{}}\PY{l+s}{female\PYZsq{}}
         women\PYZus{}mask\PY{p}{[}\PY{l+m}{1}\PY{o}{:}\PY{l+m}{10}\PY{p}{]}
\end{Verbatim}


    \begin{enumerate*}
\item FALSE
\item TRUE
\item TRUE
\item TRUE
\item FALSE
\item FALSE
\item FALSE
\item FALSE
\item TRUE
\item TRUE
\end{enumerate*}


    
    \hypertarget{duplicate-and-filter-to-create-a-model-women_survived}{%
\subparagraph{\texorpdfstring{duplicate and filter to create a model,
\texttt{women\_survived}}{duplicate and filter to create a model, women\_survived}}\label{duplicate-and-filter-to-create-a-model-women_survived}}

    \begin{Verbatim}[commandchars=\\\{\}]
{\color{incolor}In [{\color{incolor}41}]:} women\PYZus{}survived \PY{o}{=} \PY{k+kp}{rep}\PY{p}{(}no\PYZus{}survivors\PY{p}{)}
         women\PYZus{}survived\PY{p}{[}women\PYZus{}mask\PY{p}{]} \PY{o}{=} \PY{l+m}{1}
\end{Verbatim}


    \hypertarget{assess-accuracy-of-model-women_survived}{%
\subparagraph{\texorpdfstring{assess accuracy of model,
\texttt{women\_survived}}{assess accuracy of model, women\_survived}}\label{assess-accuracy-of-model-women_survived}}

    \begin{Verbatim}[commandchars=\\\{\}]
{\color{incolor}In [{\color{incolor}42}]:} accuracy\PY{p}{(}titanic\PY{o}{\PYZdl{}}Survived\PY{p}{,} women\PYZus{}survived\PY{p}{)}
\end{Verbatim}


    0.78675645342312

    
    \hypertarget{explaining-creation-of-prediction-vector}{%
\paragraph{Explaining Creation Of Prediction
Vector}\label{explaining-creation-of-prediction-vector}}

Explain in your own words the process by which the prediction vector,
\texttt{women\_survived}:

    \texttt{women\_survived} is a naive prediction vector in which is
initialized by using the no\_survivors vector (which is a vector of
zeroes) to serve as a baseline. \texttt{women\_mask} is then used to
change the female passengers' Survived value to 1, which results in all
women having survived according to this new vector.

    \hypertarget{can-another-feature-help}{%
\subsubsection{Can Another Feature
Help?}\label{can-another-feature-help}}

    \begin{Verbatim}[commandchars=\\\{\}]
{\color{incolor}In [{\color{incolor}43}]:} \PY{k+kp}{prop.table}\PY{p}{(}\PY{k+kp}{table}\PY{p}{(}titanic\PY{o}{\PYZdl{}}Survived\PY{p}{,} titanic\PY{o}{\PYZdl{}}Pclass\PY{p}{,} titanic\PY{o}{\PYZdl{}}Sex\PY{p}{)}\PY{p}{)}
\end{Verbatim}


    
    \begin{verbatim}
, ,  = female

   
              1           2           3
  0 0.003367003 0.006734007 0.080808081
  1 0.102132435 0.078563412 0.080808081

, ,  = male

   
              1           2           3
  0 0.086419753 0.102132435 0.336700337
  1 0.050505051 0.019079686 0.052749719

    \end{verbatim}

    
    \hypertarget{create-a-mask-of-just-first-class}{%
\subparagraph{create a mask of just first
class}\label{create-a-mask-of-just-first-class}}

    \begin{Verbatim}[commandchars=\\\{\}]
{\color{incolor}In [{\color{incolor}44}]:} first\PYZus{}class\PYZus{}mask \PY{o}{=} titanic\PY{o}{\PYZdl{}}Pclass \PY{o}{==} \PY{l+m}{1}
         first\PYZus{}class\PYZus{}mask\PY{p}{[}\PY{l+m}{1}\PY{o}{:}\PY{l+m}{10}\PY{p}{]}
\end{Verbatim}


    \begin{enumerate*}
\item FALSE
\item TRUE
\item FALSE
\item TRUE
\item FALSE
\item FALSE
\item TRUE
\item FALSE
\item FALSE
\item FALSE
\end{enumerate*}


    
    \hypertarget{duplicate-and-filter-to-create-a-model-women_and_first_class_survived}{%
\subparagraph{\texorpdfstring{duplicate and filter to create a model,
\texttt{women\_and\_first\_class\_survived}}{duplicate and filter to create a model, women\_and\_first\_class\_survived}}\label{duplicate-and-filter-to-create-a-model-women_and_first_class_survived}}

    \begin{Verbatim}[commandchars=\\\{\}]
{\color{incolor}In [{\color{incolor}45}]:} women\PYZus{}and\PYZus{}first\PYZus{}class\PYZus{}survived \PY{o}{=} \PY{k+kp}{rep}\PY{p}{(}women\PYZus{}survived\PY{p}{)}
         women\PYZus{}and\PYZus{}first\PYZus{}class\PYZus{}survived\PY{p}{[}first\PYZus{}class\PYZus{}mask\PY{p}{]} \PY{o}{=} \PY{l+m}{1}
\end{Verbatim}


    \hypertarget{assess-accuracy-of-model-women_and_first_class_survived}{%
\subparagraph{\texorpdfstring{assess accuracy of model,
\texttt{women\_and\_first\_class\_survived}}{assess accuracy of model, women\_and\_first\_class\_survived}}\label{assess-accuracy-of-model-women_and_first_class_survived}}

    \begin{Verbatim}[commandchars=\\\{\}]
{\color{incolor}In [{\color{incolor}46}]:} accuracy\PY{p}{(}titanic\PY{o}{\PYZdl{}}Survived\PY{p}{,} women\PYZus{}and\PYZus{}first\PYZus{}class\PYZus{}survived\PY{p}{)}
\end{Verbatim}


    0.750841750841751

    
    \begin{Verbatim}[commandchars=\\\{\}]
{\color{incolor}In [{\color{incolor}47}]:} scores \PY{o}{=} \PY{k+kt}{c}\PY{p}{(}accuracy\PY{p}{(}titanic\PY{o}{\PYZdl{}}Survived\PY{p}{,} no\PYZus{}survivors\PY{p}{)}\PY{p}{,}
                    accuracy\PY{p}{(}titanic\PY{o}{\PYZdl{}}Survived\PY{p}{,} random\PYZus{}model\PY{p}{)}\PY{p}{,}
                    accuracy\PY{p}{(}titanic\PY{o}{\PYZdl{}}Survived\PY{p}{,} women\PYZus{}survived\PY{p}{)}\PY{p}{,}
                    accuracy\PY{p}{(}titanic\PY{o}{\PYZdl{}}Survived\PY{p}{,} women\PYZus{}and\PYZus{}first\PYZus{}class\PYZus{}survived\PY{p}{)}\PY{p}{)}
\end{Verbatim}


    \hypertarget{progress-report}{%
\subsubsection{Progress Report}\label{progress-report}}

    \begin{Verbatim}[commandchars=\\\{\}]
{\color{incolor}In [{\color{incolor}48}]:} barplot\PY{p}{(}scores\PY{p}{,} xlab \PY{o}{=} \PY{l+s}{\PYZsq{}}\PY{l+s}{model\PYZsq{}}\PY{p}{,} 
                 names.arg \PY{o}{=} \PY{k+kt}{c}\PY{p}{(}\PY{l+s}{\PYZsq{}}\PY{l+s}{none\PYZsq{}}\PY{p}{,}\PY{l+s}{\PYZsq{}}\PY{l+s}{rand\PYZsq{}}\PY{p}{,}\PY{l+s}{\PYZsq{}}\PY{l+s}{women\PYZsq{}}\PY{p}{,} \PY{l+s}{\PYZsq{}}\PY{l+s}{women+1stclass\PYZsq{}}\PY{p}{)}\PY{p}{)}
         abline\PY{p}{(}h \PY{o}{=} \PY{k+kp}{max}\PY{p}{(}scores\PY{p}{)}\PY{p}{)}
\end{Verbatim}


    \begin{center}
    \adjustimage{max size={0.9\linewidth}{0.9\paperheight}}{output_135_0.png}
    \end{center}
    { \hspace*{\fill} \\}
    
    \hypertarget{age}{%
\subsubsection{Age}\label{age}}

Age is a numerical feature.

Age has missing values.

    \begin{Verbatim}[commandchars=\\\{\}]
{\color{incolor}In [{\color{incolor}49}]:} hist\PY{p}{(}titanic\PY{o}{\PYZdl{}}Age\PY{p}{)}
\end{Verbatim}


    \begin{center}
    \adjustimage{max size={0.9\linewidth}{0.9\paperheight}}{output_137_0.png}
    \end{center}
    { \hspace*{\fill} \\}
    
    \begin{Verbatim}[commandchars=\\\{\}]
{\color{incolor}In [{\color{incolor}50}]:} missing\PYZus{}age\PYZus{}values\PYZus{}mask \PY{o}{=} \PY{k+kp}{is.na}\PY{p}{(}titanic\PY{o}{\PYZdl{}}Age\PY{p}{)}
\end{Verbatim}


    \begin{Verbatim}[commandchars=\\\{\}]
{\color{incolor}In [{\color{incolor}51}]:} titanic\PY{o}{\PYZdl{}}Survived\PY{p}{[}missing\PYZus{}age\PYZus{}values\PYZus{}mask\PY{p}{]}
\end{Verbatim}


    \begin{enumerate*}
\item 0
\item 1
\item 1
\item 0
\item 1
\item 0
\item 1
\item 1
\item 1
\item 0
\item 0
\item 0
\item 1
\item 0
\item 1
\item 0
\item 1
\item 0
\item 0
\item 1
\item 0
\item 0
\item 0
\item 1
\item 1
\item 0
\item 0
\item 1
\item 0
\item 0
\item 0
\item 0
\item 1
\item 0
\item 0
\item 0
\item 0
\item 0
\item 1
\item 0
\item 1
\item 0
\item 0
\item 0
\item 0
\item 0
\item 0
\item 1
\item 0
\item 1
\item 0
\item 0
\item 0
\item 1
\item 0
\item 0
\item 0
\item 1
\item 1
\item 1
\item 1
\item 0
\item 1
\item 0
\item 1
\item 1
\item 0
\item 1
\item 0
\item 0
\item 1
\item 1
\item 0
\item 1
\item 1
\item 1
\item 0
\item 0
\item 0
\item 0
\item 0
\item 0
\item 0
\item 0
\item 0
\item 0
\item 1
\item 1
\item 0
\item 0
\item 1
\item 0
\item 0
\item 0
\item 0
\item 0
\item 0
\item 0
\item 0
\item 0
\item 0
\item 0
\item 0
\item 1
\item 0
\item 0
\item 0
\item 0
\item 0
\item 0
\item 1
\item 0
\item 1
\item 0
\item 0
\item 0
\item 0
\item 0
\item 0
\item 1
\item 0
\item 0
\item 0
\item 0
\item 1
\item 0
\item 0
\item 0
\item 0
\item 1
\item 0
\item 0
\item 0
\item 0
\item 1
\item 0
\item 0
\item 1
\item 0
\item 0
\item 1
\item 0
\item 0
\item 1
\item 1
\item 1
\item 0
\item 0
\item 1
\item 0
\item 0
\item 0
\item 1
\item 0
\item 0
\item 0
\item 0
\item 0
\item 0
\item 0
\item 0
\item 0
\item 0
\item 0
\item 0
\item 0
\item 1
\item 0
\item 0
\item 1
\item 0
\item 1
\item 0
\item 0
\item 0
\item 0
\item 0
\end{enumerate*}


    
    \begin{Verbatim}[commandchars=\\\{\}]
{\color{incolor}In [{\color{incolor}52}]:} survived\PYZus{}mask \PY{o}{=} \PY{k+kp}{as.logical}\PY{p}{(}titanic\PY{o}{\PYZdl{}}Survived\PY{p}{)}
\end{Verbatim}


    \begin{Verbatim}[commandchars=\\\{\}]
{\color{incolor}In [{\color{incolor}53}]:} h1 \PY{o}{=} hist\PY{p}{(}titanic\PY{o}{\PYZdl{}}Age\PY{p}{[}survived\PYZus{}mask\PY{p}{]}\PY{p}{,} col\PY{o}{=}rgb\PY{p}{(}\PY{l+m}{0}\PY{p}{,}\PY{l+m}{0}\PY{p}{,}\PY{l+m}{1}\PY{p}{,}\PY{l+m}{1}\PY{o}{/}\PY{l+m}{4}\PY{p}{)}\PY{p}{,} 
                   freq \PY{o}{=} \PY{n+nb+bp}{F}\PY{p}{,} breaks \PY{o}{=} \PY{l+m}{30}\PY{p}{,} ylim \PY{o}{=} \PY{k+kt}{c}\PY{p}{(}\PY{l+m}{0}\PY{p}{,}\PY{l+m}{0.04}\PY{p}{)}\PY{p}{,}
                   main\PY{o}{=}\PY{l+s}{\PYZsq{}}\PY{l+s}{Distribution Plots of Age \PYZhy{} Survived\PYZsq{}}\PY{p}{)}
         h2 \PY{o}{=} hist\PY{p}{(}titanic\PY{o}{\PYZdl{}}Age\PY{p}{[}\PY{o}{\PYZhy{}}survived\PYZus{}mask\PY{p}{]}\PY{p}{,} col\PY{o}{=}rgb\PY{p}{(}\PY{l+m}{1}\PY{p}{,}\PY{l+m}{0}\PY{p}{,}\PY{l+m}{0}\PY{p}{,}\PY{l+m}{1}\PY{o}{/}\PY{l+m}{4}\PY{p}{)}\PY{p}{,} 
                   freq \PY{o}{=} \PY{n+nb+bp}{F}\PY{p}{,} breaks \PY{o}{=} \PY{l+m}{30}\PY{p}{,} ylim \PY{o}{=} \PY{k+kt}{c}\PY{p}{(}\PY{l+m}{0}\PY{p}{,}\PY{l+m}{0.04}\PY{p}{)}\PY{p}{,}
                   main \PY{o}{=} \PY{l+s}{\PYZsq{}}\PY{l+s}{Distribution Plots of Age \PYZhy{} Did Not Survive\PYZsq{}}\PY{p}{)}
\end{Verbatim}


    \begin{center}
    \adjustimage{max size={0.9\linewidth}{0.9\paperheight}}{output_141_0.png}
    \end{center}
    { \hspace*{\fill} \\}
    
    \begin{center}
    \adjustimage{max size={0.9\linewidth}{0.9\paperheight}}{output_141_1.png}
    \end{center}
    { \hspace*{\fill} \\}
    
    \hypertarget{create-a-mask-of-just-children}{%
\subparagraph{create a mask of just
children}\label{create-a-mask-of-just-children}}

    \begin{Verbatim}[commandchars=\\\{\}]
{\color{incolor}In [{\color{incolor}54}]:} children\PYZus{}mask \PY{o}{=} titanic\PY{o}{\PYZdl{}}Age \PY{o}{\PYZlt{}} \PY{l+m}{10}
\end{Verbatim}


    \hypertarget{duplicate-and-filter-to-create-a-model-women_survived}{%
\subparagraph{\texorpdfstring{duplicate and filter to create a model,
\texttt{women\_survived}}{duplicate and filter to create a model, women\_survived}}\label{duplicate-and-filter-to-create-a-model-women_survived}}

    \begin{Verbatim}[commandchars=\\\{\}]
{\color{incolor}In [{\color{incolor}55}]:} women\PYZus{}and\PYZus{}children\PYZus{}survived \PY{o}{=} \PY{k+kp}{rep}\PY{p}{(}women\PYZus{}survived\PY{p}{)}
         women\PYZus{}and\PYZus{}children\PYZus{}survived\PY{p}{[}children\PYZus{}mask\PY{p}{]} \PY{o}{=} \PY{l+m}{1}
\end{Verbatim}


    \hypertarget{assess-accuracy-of-model-women_survived}{%
\subparagraph{\texorpdfstring{assess accuracy of model,
\texttt{women\_survived}}{assess accuracy of model, women\_survived}}\label{assess-accuracy-of-model-women_survived}}

    \begin{Verbatim}[commandchars=\\\{\}]
{\color{incolor}In [{\color{incolor}56}]:} accuracy\PY{p}{(}titanic\PY{o}{\PYZdl{}}Survived\PY{p}{,} women\PYZus{}and\PYZus{}children\PYZus{}survived\PY{p}{)}
\end{Verbatim}


    0.793490460157127

    
    \begin{Verbatim}[commandchars=\\\{\}]
{\color{incolor}In [{\color{incolor}57}]:} scores \PY{o}{=} \PY{k+kt}{c}\PY{p}{(}accuracy\PY{p}{(}titanic\PY{o}{\PYZdl{}}Survived\PY{p}{,} no\PYZus{}survivors\PY{p}{)}\PY{p}{,}
                    accuracy\PY{p}{(}titanic\PY{o}{\PYZdl{}}Survived\PY{p}{,} random\PYZus{}model\PY{p}{)}\PY{p}{,}
                    accuracy\PY{p}{(}titanic\PY{o}{\PYZdl{}}Survived\PY{p}{,} women\PYZus{}survived\PY{p}{)}\PY{p}{,}
                    accuracy\PY{p}{(}titanic\PY{o}{\PYZdl{}}Survived\PY{p}{,} women\PYZus{}and\PYZus{}first\PYZus{}class\PYZus{}survived\PY{p}{)}\PY{p}{,}
                    accuracy\PY{p}{(}titanic\PY{o}{\PYZdl{}}Survived\PY{p}{,} women\PYZus{}and\PYZus{}children\PYZus{}survived\PY{p}{)}\PY{p}{)}
\end{Verbatim}


    \hypertarget{progress-report}{%
\subsubsection{Progress Report}\label{progress-report}}

    \begin{Verbatim}[commandchars=\\\{\}]
{\color{incolor}In [{\color{incolor}59}]:} barplot\PY{p}{(}scores\PY{p}{,} xlab \PY{o}{=} \PY{l+s}{\PYZsq{}}\PY{l+s}{model\PYZsq{}}\PY{p}{,} 
                 names.arg \PY{o}{=} \PY{k+kt}{c}\PY{p}{(}\PY{l+s}{\PYZsq{}}\PY{l+s}{none\PYZsq{}}\PY{p}{,}\PY{l+s}{\PYZsq{}}\PY{l+s}{rand\PYZsq{}}\PY{p}{,}\PY{l+s}{\PYZsq{}}\PY{l+s}{women\PYZsq{}}\PY{p}{,} \PY{l+s}{\PYZsq{}}\PY{l+s}{women+1stclass\PYZsq{}}\PY{p}{,} \PY{l+s}{\PYZsq{}}\PY{l+s}{women+children\PYZsq{}}\PY{p}{)}\PY{p}{)}
         abline\PY{p}{(}h \PY{o}{=} \PY{k+kp}{max}\PY{p}{(}scores\PY{p}{)}\PY{p}{)}
\end{Verbatim}


    \begin{center}
    \adjustimage{max size={0.9\linewidth}{0.9\paperheight}}{output_150_0.png}
    \end{center}
    { \hspace*{\fill} \\}
    
    \begin{Verbatim}[commandchars=\\\{\}]
{\color{incolor}In [{\color{incolor}60}]:} child\PYZus{}survival\PYZus{}by\PYZus{}age \PY{o}{=} \PY{k+kr}{function} \PY{p}{(}age\PY{p}{)} \PY{p}{\PYZob{}}
             children\PYZus{}mask \PY{o}{=} titanic\PY{o}{\PYZdl{}}Age \PY{o}{\PYZlt{}} age
         
             women\PYZus{}and\PYZus{}children\PYZus{}survived \PY{o}{=} \PY{k+kp}{rep}\PY{p}{(}women\PYZus{}survived\PY{p}{)}
             women\PYZus{}and\PYZus{}children\PYZus{}survived\PY{p}{[}children\PYZus{}mask\PY{p}{]} \PY{o}{=} \PY{l+m}{1}
         
             \PY{k+kr}{return}\PY{p}{(}accuracy\PY{p}{(}titanic\PY{o}{\PYZdl{}}Survived\PY{p}{,} women\PYZus{}and\PYZus{}children\PYZus{}survived\PY{p}{)}\PY{p}{)}
         \PY{p}{\PYZcb{}}
\end{Verbatim}


    \begin{Verbatim}[commandchars=\\\{\}]
{\color{incolor}In [{\color{incolor}61}]:} ages \PY{o}{=} \PY{l+m}{1}\PY{o}{:}\PY{l+m}{20}
\end{Verbatim}


    \begin{Verbatim}[commandchars=\\\{\}]
{\color{incolor}In [{\color{incolor}63}]:} plot\PY{p}{(}\PY{k+kp}{sapply}\PY{p}{(}ages\PY{p}{,} child\PYZus{}survival\PYZus{}by\PYZus{}age\PY{p}{)}\PY{p}{)}
\end{Verbatim}


    \begin{center}
    \adjustimage{max size={0.9\linewidth}{0.9\paperheight}}{output_153_0.png}
    \end{center}
    { \hspace*{\fill} \\}
    
    \hypertarget{create-a-mask-of-just-children}{%
\subparagraph{create a mask of just
children}\label{create-a-mask-of-just-children}}

    \begin{Verbatim}[commandchars=\\\{\}]
{\color{incolor}In [{\color{incolor}64}]:} children\PYZus{}mask \PY{o}{=} titanic\PY{o}{\PYZdl{}}Age \PY{o}{\PYZlt{}} \PY{l+m}{7}
\end{Verbatim}


    \hypertarget{duplicate-and-filter-to-create-a-model-women_survived}{%
\subparagraph{\texorpdfstring{duplicate and filter to create a model,
\texttt{women\_survived}}{duplicate and filter to create a model, women\_survived}}\label{duplicate-and-filter-to-create-a-model-women_survived}}

    \begin{Verbatim}[commandchars=\\\{\}]
{\color{incolor}In [{\color{incolor}65}]:} women\PYZus{}and\PYZus{}children\PYZus{}survived \PY{o}{=} \PY{k+kp}{rep}\PY{p}{(}women\PYZus{}survived\PY{p}{)}
         women\PYZus{}and\PYZus{}children\PYZus{}survived\PY{p}{[}children\PYZus{}mask\PY{p}{]} \PY{o}{=} \PY{l+m}{1}
\end{Verbatim}


    \hypertarget{assess-accuracy-of-model-women_survived}{%
\subparagraph{\texorpdfstring{assess accuracy of model,
\texttt{women\_survived}}{assess accuracy of model, women\_survived}}\label{assess-accuracy-of-model-women_survived}}

    \begin{Verbatim}[commandchars=\\\{\}]
{\color{incolor}In [{\color{incolor}66}]:} accuracy\PY{p}{(}titanic\PY{o}{\PYZdl{}}Survived\PY{p}{,} women\PYZus{}and\PYZus{}children\PYZus{}survived\PY{p}{)}
\end{Verbatim}


    0.795735129068462

    
    \begin{Verbatim}[commandchars=\\\{\}]
{\color{incolor}In [{\color{incolor}67}]:} scores \PY{o}{=} \PY{k+kt}{c}\PY{p}{(}accuracy\PY{p}{(}titanic\PY{o}{\PYZdl{}}Survived\PY{p}{,} no\PYZus{}survivors\PY{p}{)}\PY{p}{,}
                    accuracy\PY{p}{(}titanic\PY{o}{\PYZdl{}}Survived\PY{p}{,} random\PYZus{}model\PY{p}{)}\PY{p}{,}
                    accuracy\PY{p}{(}titanic\PY{o}{\PYZdl{}}Survived\PY{p}{,} women\PYZus{}survived\PY{p}{)}\PY{p}{,}
                    accuracy\PY{p}{(}titanic\PY{o}{\PYZdl{}}Survived\PY{p}{,} women\PYZus{}and\PYZus{}first\PYZus{}class\PYZus{}survived\PY{p}{)}\PY{p}{,}
                    accuracy\PY{p}{(}titanic\PY{o}{\PYZdl{}}Survived\PY{p}{,} women\PYZus{}and\PYZus{}children\PYZus{}survived\PY{p}{)}\PY{p}{)}
\end{Verbatim}


    \hypertarget{progress-report}{%
\subsubsection{Progress Report}\label{progress-report}}

    \begin{Verbatim}[commandchars=\\\{\}]
{\color{incolor}In [{\color{incolor}68}]:} barplot\PY{p}{(}scores\PY{p}{,} xlab \PY{o}{=} \PY{l+s}{\PYZsq{}}\PY{l+s}{model\PYZsq{}}\PY{p}{,} 
                 names.arg \PY{o}{=} \PY{k+kt}{c}\PY{p}{(}\PY{l+s}{\PYZsq{}}\PY{l+s}{none\PYZsq{}}\PY{p}{,}\PY{l+s}{\PYZsq{}}\PY{l+s}{rand\PYZsq{}}\PY{p}{,}\PY{l+s}{\PYZsq{}}\PY{l+s}{women\PYZsq{}}\PY{p}{,} \PY{l+s}{\PYZsq{}}\PY{l+s}{women+1stclass\PYZsq{}}\PY{p}{,} \PY{l+s}{\PYZsq{}}\PY{l+s}{women+children\PYZsq{}}\PY{p}{)}\PY{p}{)}
         abline\PY{p}{(}h \PY{o}{=} \PY{k+kp}{max}\PY{p}{(}scores\PY{p}{)}\PY{p}{)}
\end{Verbatim}


    \begin{center}
    \adjustimage{max size={0.9\linewidth}{0.9\paperheight}}{output_162_0.png}
    \end{center}
    { \hspace*{\fill} \\}
    

    % Add a bibliography block to the postdoc
    
    
    
    \end{document}
